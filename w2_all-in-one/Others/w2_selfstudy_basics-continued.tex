\documentclass[]{article}
\usepackage{lmodern}
\usepackage{amssymb,amsmath}
\usepackage{ifxetex,ifluatex}
\usepackage{fixltx2e} % provides \textsubscript
\ifnum 0\ifxetex 1\fi\ifluatex 1\fi=0 % if pdftex
  \usepackage[T1]{fontenc}
  \usepackage[utf8]{inputenc}
\else % if luatex or xelatex
  \ifxetex
    \usepackage{mathspec}
  \else
    \usepackage{fontspec}
  \fi
  \defaultfontfeatures{Ligatures=TeX,Scale=MatchLowercase}
\fi
% use upquote if available, for straight quotes in verbatim environments
\IfFileExists{upquote.sty}{\usepackage{upquote}}{}
% use microtype if available
\IfFileExists{microtype.sty}{%
\usepackage{microtype}
\UseMicrotypeSet[protrusion]{basicmath} % disable protrusion for tt fonts
}{}
\usepackage[margin=1in]{geometry}
\usepackage{hyperref}
\hypersetup{unicode=true,
            pdftitle={SCR week 2: homework exercises},
            pdfborder={0 0 0},
            breaklinks=true}
\urlstyle{same}  % don't use monospace font for urls
\usepackage{color}
\usepackage{fancyvrb}
\newcommand{\VerbBar}{|}
\newcommand{\VERB}{\Verb[commandchars=\\\{\}]}
\DefineVerbatimEnvironment{Highlighting}{Verbatim}{commandchars=\\\{\}}
% Add ',fontsize=\small' for more characters per line
\usepackage{framed}
\definecolor{shadecolor}{RGB}{248,248,248}
\newenvironment{Shaded}{\begin{snugshade}}{\end{snugshade}}
\newcommand{\AlertTok}[1]{\textcolor[rgb]{0.94,0.16,0.16}{#1}}
\newcommand{\AnnotationTok}[1]{\textcolor[rgb]{0.56,0.35,0.01}{\textbf{\textit{#1}}}}
\newcommand{\AttributeTok}[1]{\textcolor[rgb]{0.77,0.63,0.00}{#1}}
\newcommand{\BaseNTok}[1]{\textcolor[rgb]{0.00,0.00,0.81}{#1}}
\newcommand{\BuiltInTok}[1]{#1}
\newcommand{\CharTok}[1]{\textcolor[rgb]{0.31,0.60,0.02}{#1}}
\newcommand{\CommentTok}[1]{\textcolor[rgb]{0.56,0.35,0.01}{\textit{#1}}}
\newcommand{\CommentVarTok}[1]{\textcolor[rgb]{0.56,0.35,0.01}{\textbf{\textit{#1}}}}
\newcommand{\ConstantTok}[1]{\textcolor[rgb]{0.00,0.00,0.00}{#1}}
\newcommand{\ControlFlowTok}[1]{\textcolor[rgb]{0.13,0.29,0.53}{\textbf{#1}}}
\newcommand{\DataTypeTok}[1]{\textcolor[rgb]{0.13,0.29,0.53}{#1}}
\newcommand{\DecValTok}[1]{\textcolor[rgb]{0.00,0.00,0.81}{#1}}
\newcommand{\DocumentationTok}[1]{\textcolor[rgb]{0.56,0.35,0.01}{\textbf{\textit{#1}}}}
\newcommand{\ErrorTok}[1]{\textcolor[rgb]{0.64,0.00,0.00}{\textbf{#1}}}
\newcommand{\ExtensionTok}[1]{#1}
\newcommand{\FloatTok}[1]{\textcolor[rgb]{0.00,0.00,0.81}{#1}}
\newcommand{\FunctionTok}[1]{\textcolor[rgb]{0.00,0.00,0.00}{#1}}
\newcommand{\ImportTok}[1]{#1}
\newcommand{\InformationTok}[1]{\textcolor[rgb]{0.56,0.35,0.01}{\textbf{\textit{#1}}}}
\newcommand{\KeywordTok}[1]{\textcolor[rgb]{0.13,0.29,0.53}{\textbf{#1}}}
\newcommand{\NormalTok}[1]{#1}
\newcommand{\OperatorTok}[1]{\textcolor[rgb]{0.81,0.36,0.00}{\textbf{#1}}}
\newcommand{\OtherTok}[1]{\textcolor[rgb]{0.56,0.35,0.01}{#1}}
\newcommand{\PreprocessorTok}[1]{\textcolor[rgb]{0.56,0.35,0.01}{\textit{#1}}}
\newcommand{\RegionMarkerTok}[1]{#1}
\newcommand{\SpecialCharTok}[1]{\textcolor[rgb]{0.00,0.00,0.00}{#1}}
\newcommand{\SpecialStringTok}[1]{\textcolor[rgb]{0.31,0.60,0.02}{#1}}
\newcommand{\StringTok}[1]{\textcolor[rgb]{0.31,0.60,0.02}{#1}}
\newcommand{\VariableTok}[1]{\textcolor[rgb]{0.00,0.00,0.00}{#1}}
\newcommand{\VerbatimStringTok}[1]{\textcolor[rgb]{0.31,0.60,0.02}{#1}}
\newcommand{\WarningTok}[1]{\textcolor[rgb]{0.56,0.35,0.01}{\textbf{\textit{#1}}}}
\usepackage{graphicx,grffile}
\makeatletter
\def\maxwidth{\ifdim\Gin@nat@width>\linewidth\linewidth\else\Gin@nat@width\fi}
\def\maxheight{\ifdim\Gin@nat@height>\textheight\textheight\else\Gin@nat@height\fi}
\makeatother
% Scale images if necessary, so that they will not overflow the page
% margins by default, and it is still possible to overwrite the defaults
% using explicit options in \includegraphics[width, height, ...]{}
\setkeys{Gin}{width=\maxwidth,height=\maxheight,keepaspectratio}
\IfFileExists{parskip.sty}{%
\usepackage{parskip}
}{% else
\setlength{\parindent}{0pt}
\setlength{\parskip}{6pt plus 2pt minus 1pt}
}
\setlength{\emergencystretch}{3em}  % prevent overfull lines
\providecommand{\tightlist}{%
  \setlength{\itemsep}{0pt}\setlength{\parskip}{0pt}}
\setcounter{secnumdepth}{0}
% Redefines (sub)paragraphs to behave more like sections
\ifx\paragraph\undefined\else
\let\oldparagraph\paragraph
\renewcommand{\paragraph}[1]{\oldparagraph{#1}\mbox{}}
\fi
\ifx\subparagraph\undefined\else
\let\oldsubparagraph\subparagraph
\renewcommand{\subparagraph}[1]{\oldsubparagraph{#1}\mbox{}}
\fi

%%% Use protect on footnotes to avoid problems with footnotes in titles
\let\rmarkdownfootnote\footnote%
\def\footnote{\protect\rmarkdownfootnote}

%%% Change title format to be more compact
\usepackage{titling}

% Create subtitle command for use in maketitle
\providecommand{\subtitle}[1]{
  \posttitle{
    \begin{center}\large#1\end{center}
    }
}

\setlength{\droptitle}{-2em}

  \title{SCR week 2: homework exercises}
    \pretitle{\vspace{\droptitle}\centering\huge}
  \posttitle{\par}
    \author{}
    \preauthor{}\postauthor{}
    \date{}
    \predate{}\postdate{}
  
\usepackage{graphicx}
\usepackage{float}
\usepackage{placeins}

\begin{document}
\maketitle

The best way to master the \texttt{R} basics is to code yourself and
\texttt{rep("practice",\ Inf)}. A good strategy to work on these
exercises is to do them together with your colleagues, and discuss your
(possibly) different strategies and solutions to the exercises. A few of
the exercises are mandatory: they talk about some new things we've not
covered in class, but which you need to know.

The other part is optional: these provide you with more material to
practice your skills. These are of course highly recommended.

\hypertarget{packages-data-and-more}{%
\subsection{1. Packages, data, and more}\label{packages-data-and-more}}

\hypertarget{a.}{%
\subsubsection{a.}\label{a.}}

Install and load the package \texttt{nycflights13}. Take a look at the
\texttt{nycflights13} package and \texttt{flights} object documentation
using
\texttt{help(package\ =\ \textquotesingle{}nycflights13\textquotesingle{})}
and take a look at the \texttt{flights} object, using
\texttt{data(flights)}. Note: the \texttt{flights} data, is in a
\texttt{tibble} format: a class similar to data.frame (something that
differs is the way the data frame is \texttt{print}ed for example). It
is enough to know you can interact with a \texttt{tibble} in mostly the
same way as a data.frame. Try some things you know can do with
data.frames on the \texttt{flights} object, to see this for yourself.

\hypertarget{b.}{%
\subsubsection{b.}\label{b.}}

Having read the documentation, explore the data (the \texttt{flights}
object) by using functions like \texttt{class}, \texttt{dim},
\texttt{head}, \texttt{str} and \texttt{summary}.

\hypertarget{c.}{%
\subsubsection{c.}\label{c.}}

Let's do some basic filtering. Create logical vectors such that we can
find all flights:

\begin{itemize}
\tightlist
\item
  to SFO or OAK\\
\item
  delayed by more than an hour\\
\item
  that departed between midnight (including midnight) and 5.00 am\\
\item
  for which the arrival delay was more than twice the departure delay
\end{itemize}

\hypertarget{d.}{%
\subsubsection{d.}\label{d.}}

Combine the logical vectors to see if there are any flights for which
\emph{all} of the above are true. Are there any?

\hypertarget{creating-variables-and-seed}{%
\subsection{\texorpdfstring{2. Creating variables and
\texttt{seed}}{2. Creating variables and seed}}\label{creating-variables-and-seed}}

First run the following code to create a data frame \texttt{data.set}
with no variables:

\begin{Shaded}
\begin{Highlighting}[]
\KeywordTok{library}\NormalTok{(}\StringTok{"nycflights13"}\NormalTok{)}
\KeywordTok{data}\NormalTok{(flights)}

\KeywordTok{set.seed}\NormalTok{(}\DecValTok{20161013}\NormalTok{)}

\NormalTok{N <-}\StringTok{ }\FloatTok{1e3}
\NormalTok{indx.flights <-}\StringTok{ }\KeywordTok{sample}\NormalTok{(}\DecValTok{1}\OperatorTok{:}\KeywordTok{nrow}\NormalTok{(flights), N) }
\NormalTok{data.set <-}\StringTok{ }\KeywordTok{data.frame}\NormalTok{(}\DataTypeTok{row.names =} \DecValTok{1}\OperatorTok{:}\NormalTok{N)}
\end{Highlighting}
\end{Shaded}

Now, we would like to add the following variables.

\begin{itemize}
\tightlist
\item
  \texttt{x1}: numeric vector with components
  \(N/2,N/2 -1, N/2 - 2,\ldots,2,1,1,2 \ldots, N/2 - 2, N/2 - 1,N /2\).
\item
  \texttt{x2}: a logical vector of lenght \texttt{N}. Extract the
  carrier information out of the \texttt{flights} data for the
  rownumbers that can be found in \texttt{indx.flights}. In the logical
  vector give each element that has `US',`UA' or `AA' in the
  corresponding position in the carrier information vector a
  \texttt{TRUE}, all others \texttt{FALSE}. \emph{HINT: using the binary
  operator \texttt{\%in\%} may be convenient for you}
\item
  \texttt{x3}: a `bad weather' indication variable. Draw \(N\)
  observations from a standard normal distribution, then square and
  round the results to the nearest integer (use the functions
  \texttt{rnorm} and \texttt{round}).
\item
  \texttt{e}: a vector of \(N\) elements that come from a standard
  normal i.i.d.
\item
  \texttt{y}: our own synthetic outcome variable representing arrival
  delay: \(y_i = 2 + 0.1*x_{i2} + 2*x_{i3} + 1*x_{i2}*x_{i3} + e_i\).
\end{itemize}

\hypertarget{b.-1}{%
\subsubsection{b.}\label{b.-1}}

Write the data to a file called \texttt{my\_fake\_data.csv} in the
\texttt{0\_data} folder.

\hypertarget{c.-1}{%
\subsubsection{c.}\label{c.-1}}

Share you file with a fellow student and have them share their file with
yours (e.g.~send it via e-mail). Are they different? If so, how? Are
they the same in surprising places? How about all the random numbers,
are they different?

\textbf{Answer:} If you run the exact same code, given that we set a
seed at the start of this exercise, you should get the exact same random
numbers. However, if you've samped some random values in between final
results, the first results might be the same, but the latter won't be.

\hypertarget{outro}{%
\subsubsection{Outro}\label{outro}}

We'll talk more about seeds (and \texttt{set.seed}) in a later lecture.
For now it is enough to realise that setting a seed, means putting the
random number generator of the computer into a particular state: you
make sure two people get the same random numbers, by making sure they
set the state of the generator in the same way, using \texttt{set.seed}.

\hypertarget{working-with-vectors}{%
\subsection{3. Working with vectors}\label{working-with-vectors}}

Given two vectors:

\begin{Shaded}
\begin{Highlighting}[]
\NormalTok{x <-}\StringTok{ }\KeywordTok{c}\NormalTok{(}\DecValTok{5}\NormalTok{, }\DecValTok{2}\NormalTok{, }\DecValTok{10}\NormalTok{, }\DecValTok{4}\NormalTok{) }
\NormalTok{y <-}\StringTok{ }\KeywordTok{c}\NormalTok{(}\DecValTok{3}\NormalTok{, }\DecValTok{6}\NormalTok{, }\DecValTok{3}\NormalTok{, }\DecValTok{10}\NormalTok{)}
\end{Highlighting}
\end{Shaded}

Using one (or some) of the operators \texttt{\&}, \texttt{\&\&},
\texttt{\textbar{}}, \texttt{\textbar{}\textbar{}},
\texttt{\textgreater{}}, \texttt{any()}, \texttt{all()}, and write
\texttt{R} code to test if:

\begin{enumerate}
\def\labelenumi{\alph{enumi}.}
\tightlist
\item
  elements in vector \texttt{x} are greater than elements in vector
  \texttt{y};\\
\item
  elements in vectors \texttt{x} AND \texttt{y} are greater than 3;\\
\item
  elements in vectors \texttt{x} OR \texttt{y} are greater than 3;\\
\item
  all elements of vector \texttt{x} AND all elements of vector
  \texttt{y} are greater than 2;\\
\item
  all elements of vector \texttt{x} OR all elements of vector \texttt{y}
  are greater than 2;\\
\item
  there are any elements in vectors \texttt{x} or \texttt{y} that are
  greater than 9;\\
\item
  the first element of vector \texttt{x} AND the first element of vector
  \texttt{y} are greater than 2
\end{enumerate}

\hypertarget{paste-and-vector-recycling}{%
\subsection{\texorpdfstring{4. \texttt{paste} and vector
recycling}{4. paste and vector recycling}}\label{paste-and-vector-recycling}}

\hypertarget{a.-1}{%
\subsubsection{a.}\label{a.-1}}

Explore the helpfile and examples of the function \texttt{paste} if you
are not comfortable with the paste function yet.

\hypertarget{b.-2}{%
\subsubsection{b.}\label{b.-2}}

Create the vector ``varname'' which has the following elements:

\begin{verbatim}
    "A_1"  "A_2"  "A_3"  "A_4"  "A_5"  "A_6"  "A_7"  "A_8"  "A_9"  "A_10"
\end{verbatim}

\hypertarget{c.-2}{%
\subsubsection{c.}\label{c.-2}}

Remember vector recycling? Use the function \texttt{paste} and vector
recycling for \texttt{c("ODD",\ "EVEN")} to create a vector that
contains the following elements:

\begin{verbatim}
"1 = ODD" "2 = EVEN" "3 = ODD" "4 = EVEN" "5 = ODD" "6 = EVEN" 
\end{verbatim}

\hypertarget{enter-the-matrix}{%
\subsection{\texorpdfstring{5. Enter the
\texttt{matrix}}{5. Enter the matrix}}\label{enter-the-matrix}}

\hypertarget{a.-2}{%
\subsubsection{a.}\label{a.-2}}

Work through the extended example in 3.2.3 (from p.~63) about an image
of Mount Rushmore. You can find the image file (\texttt{mtrush1.pgm}) in
the \texttt{0\_data} folder. You can find the function blurpart below
(this is the corrected version of the downloaded code from the book).

\begin{Shaded}
\begin{Highlighting}[]
\NormalTok{blurpart <-}\StringTok{ }\ControlFlowTok{function}\NormalTok{(img, rows, cols, q) \{}
\NormalTok{  lrows <-}\StringTok{ }\KeywordTok{length}\NormalTok{(rows)}
\NormalTok{  lcols <-}\StringTok{ }\KeywordTok{length}\NormalTok{(cols)}
\NormalTok{  newimg <-}\StringTok{ }\NormalTok{img}
\NormalTok{  randomnoise <-}\StringTok{ }\KeywordTok{matrix}\NormalTok{(}\DataTypeTok{nrow =}\NormalTok{ lrows, }\DataTypeTok{ncol =}\NormalTok{ lcols, }\KeywordTok{runif}\NormalTok{(lrows }\OperatorTok{*}\StringTok{ }\NormalTok{lcols))}
\NormalTok{  newimg}\OperatorTok{@}\NormalTok{grey[rows, cols] <-}\StringTok{ }\NormalTok{(}\DecValTok{1} \OperatorTok{-}\StringTok{ }\NormalTok{q) }\OperatorTok{*}\StringTok{ }\NormalTok{img}\OperatorTok{@}\NormalTok{grey[rows, cols] }\OperatorTok{+}\StringTok{ }\NormalTok{q }\OperatorTok{*}\StringTok{ }\NormalTok{randomnoise}
  \KeywordTok{return}\NormalTok{(newimg)}
\NormalTok{\}}
\end{Highlighting}
\end{Shaded}

Plot the variable (= objects) \texttt{mtrush1}, \texttt{mtrush2}, and
\texttt{mtrush3} using e.g. \texttt{plot(mtrush1)}.

\hypertarget{b.-3}{%
\subsubsection{b.}\label{b.-3}}

Create object \texttt{mtrush4} using a different value of \texttt{q}.
What happens if \texttt{q} is close to \texttt{0}? and if \texttt{q} is
close to \texttt{1}?

\hypertarget{c.-3}{%
\subsubsection{c.}\label{c.-3}}

Now we want to keep President Roosevelt, but disguise the person on the
left of the figure (Hint: the index of the rows equals 25:86; the index
of the columns starts at 15).

Construct a function that works similarly to blurpart, but instead of
replacing the pixels with random noise, changes the pixels so that the
indicated pixels are really blurred.

\emph{Hint: in a blurred image, we lose all the `sharp' edges, or that
we lose contrast. Contrast is given by very light, or very dark pixels,
a method of blurring might therefore be to `pull' pixels to the average
pixel intensity, and the more intense ones (towards light or dark) more
heavily).}

Construct the object (using the function blurpart) and plot it.

\hypertarget{set.seed-and-coding-mini-simulation-studies}{%
\subsection{\texorpdfstring{6. \texttt{set.seed} and coding mini
simulation
studies}{6. set.seed and coding mini simulation studies}}\label{set.seed-and-coding-mini-simulation-studies}}

This is an exercise about generating random normally distributed vectors
and applying functions to rows and columns of a matrix.

Let matrix \(\mathbf{X}\) be a matrix of 4 columns with normally
distributed random variables with \(\mu=1\), \(\sigma^2=3\), and number
of \(N\) observations that is equal to \(5000\). Below, we show four
ways to generate \(\bf{X}\), the generated matrices are denoted with
\texttt{X1}, \texttt{X2}, \texttt{X3} and \texttt{X4}, respectively.

\begin{Shaded}
\begin{Highlighting}[]
\KeywordTok{set.seed}\NormalTok{(}\DecValTok{123}\NormalTok{)}
\NormalTok{X1 <-}\StringTok{ }\KeywordTok{cbind}\NormalTok{(}
  \KeywordTok{rnorm}\NormalTok{(}\DataTypeTok{n =} \DecValTok{5000}\NormalTok{, }\DataTypeTok{mean =} \DecValTok{1}\NormalTok{, }\DataTypeTok{sd =} \KeywordTok{sqrt}\NormalTok{(}\DecValTok{3}\NormalTok{)), }
  \KeywordTok{rnorm}\NormalTok{(}\DataTypeTok{n =} \DecValTok{5000}\NormalTok{, }\DataTypeTok{mean =} \DecValTok{1}\NormalTok{, }\DataTypeTok{sd =} \KeywordTok{sqrt}\NormalTok{(}\DecValTok{3}\NormalTok{)),}
  \KeywordTok{rnorm}\NormalTok{(}\DataTypeTok{n =} \DecValTok{5000}\NormalTok{, }\DataTypeTok{mean =} \DecValTok{1}\NormalTok{, }\DataTypeTok{sd =} \KeywordTok{sqrt}\NormalTok{(}\DecValTok{3}\NormalTok{)), }
  \KeywordTok{rnorm}\NormalTok{(}\DataTypeTok{n =} \DecValTok{5000}\NormalTok{, }\DataTypeTok{mean =} \DecValTok{1}\NormalTok{, }\DataTypeTok{sd =} \KeywordTok{sqrt}\NormalTok{(}\DecValTok{3}\NormalTok{))}
\NormalTok{) }
\end{Highlighting}
\end{Shaded}

\begin{Shaded}
\begin{Highlighting}[]
\KeywordTok{set.seed}\NormalTok{(}\DecValTok{123}\NormalTok{)}
\NormalTok{X2 <-}\StringTok{ }\KeywordTok{matrix}\NormalTok{(}
  \KeywordTok{rep}\NormalTok{(}\KeywordTok{rnorm}\NormalTok{(}\DataTypeTok{n =} \DecValTok{5000}\NormalTok{, }\DataTypeTok{mean =} \DecValTok{1}\NormalTok{, }\DataTypeTok{sd =} \KeywordTok{sqrt}\NormalTok{(}\DecValTok{3}\NormalTok{)), }\DataTypeTok{times =} \DecValTok{4}\NormalTok{),}
  \DataTypeTok{nrow =} \DecValTok{5000}\NormalTok{, }\DataTypeTok{ncol =} \DecValTok{4}
\NormalTok{)}
\end{Highlighting}
\end{Shaded}

\begin{Shaded}
\begin{Highlighting}[]
\KeywordTok{set.seed}\NormalTok{(}\DecValTok{123}\NormalTok{)}
\NormalTok{nsamples <-}\StringTok{ }\DecValTok{4}
\NormalTok{X3 <-}\StringTok{ }\KeywordTok{matrix}\NormalTok{(}
  \KeywordTok{rnorm}\NormalTok{(}\DataTypeTok{n =} \DecValTok{5000} \OperatorTok{*}\StringTok{ }\NormalTok{nsamples, }\DataTypeTok{mean =} \DecValTok{1}\NormalTok{, }\DataTypeTok{sd =} \KeywordTok{sqrt}\NormalTok{(}\DecValTok{3}\NormalTok{)), }
  \DataTypeTok{nrow =} \DecValTok{5000}
\NormalTok{)}
\end{Highlighting}
\end{Shaded}

\begin{Shaded}
\begin{Highlighting}[]
\KeywordTok{set.seed}\NormalTok{(}\DecValTok{123}\NormalTok{)}
\NormalTok{nsamples <-}\StringTok{ }\DecValTok{4}
\NormalTok{X4 <-}\StringTok{ }\KeywordTok{replicate}\NormalTok{(nsamples, \{}
  \KeywordTok{rnorm}\NormalTok{(}\DataTypeTok{n =} \DecValTok{5000}\NormalTok{, }\DataTypeTok{mean =} \DecValTok{1}\NormalTok{, }\DataTypeTok{sd =} \KeywordTok{sqrt}\NormalTok{(}\DecValTok{3}\NormalTok{))}
\NormalTok{\})}
\end{Highlighting}
\end{Shaded}

\hypertarget{a.-3}{%
\subsubsection{a.}\label{a.-3}}

Compare the column means and variances of the four matrices. Which two
matrices have the same solution? Which matrix is not a matrix with four
random variables? How come? Which coding example do you prefer and why?

\hypertarget{b.-4}{%
\subsubsection{b.}\label{b.-4}}

From matrix \texttt{X3}, create a new matrix \texttt{X3\_new} that
contains only the rows of \texttt{X3} for which at least 2 out of the 4
values are greater than 1. Give the dimensions of \texttt{X3\_new}.
Think back on logical indexing and filtering!

\hypertarget{c.-4}{%
\subsubsection{c.}\label{c.-4}}

Create a new matrix \texttt{X5} that is based on \texttt{X3}, where each
column of \texttt{X3} is standardized (i.e.~column mean equals \(0\) and
standard deviation equals \(1\)).

\hypertarget{matrices-of-character-and-numeric-mode}{%
\subsection{\texorpdfstring{8. Matrices of \texttt{character} and
\texttt{numeric}
mode}{8. Matrices of character and numeric mode}}\label{matrices-of-character-and-numeric-mode}}

Suppose we have the following matrix:

\begin{Shaded}
\begin{Highlighting}[]
\NormalTok{mat_values <-}\StringTok{ }\KeywordTok{c}\NormalTok{(}
  \KeywordTok{rep}\NormalTok{(}\DecValTok{2}\NormalTok{, }\DecValTok{3}\NormalTok{), }
  \KeywordTok{rep}\NormalTok{(}\DecValTok{3}\NormalTok{, }\DecValTok{2}\NormalTok{), }
  \DecValTok{2}\NormalTok{, }\DecValTok{3}\OperatorTok{:}\DecValTok{1}\NormalTok{, }
  \KeywordTok{rep}\NormalTok{(}\DecValTok{0}\NormalTok{, }\DecValTok{3}\NormalTok{)}
\NormalTok{)}
\NormalTok{mat <-}\StringTok{ }\KeywordTok{matrix}\NormalTok{(mat_values, }\DataTypeTok{nrow =} \DecValTok{4}
\NormalTok{)}
\NormalTok{mat}
\end{Highlighting}
\end{Shaded}

\begin{verbatim}
##      [,1] [,2] [,3]
## [1,]    2    3    1
## [2,]    2    2    0
## [3,]    2    3    0
## [4,]    3    2    0
\end{verbatim}

and the following character vector:

\begin{Shaded}
\begin{Highlighting}[]
\NormalTok{strg <-}\StringTok{ }\KeywordTok{c}\NormalTok{(}\StringTok{"A"}\NormalTok{ ,}\StringTok{"C"}\NormalTok{, }\StringTok{"G"}\NormalTok{, }\StringTok{"T"}\NormalTok{)}
\NormalTok{strg}
\end{Highlighting}
\end{Shaded}

\begin{verbatim}
## [1] "A" "C" "G" "T"
\end{verbatim}

The values in matrix mat correspond to the characters of the vector
`strg' in the following way: the value 0 corresponds to A (Adenine), the
value 1 corresponds to C (Cytosine), the value 2 corresponds to G
(Guanine), and the value 3 corresponds to T (Thymine).

\hypertarget{a.-4}{%
\subsubsection{a.}\label{a.-4}}

Write a piece of \texttt{R} code that converts each value of
\texttt{mat} into the corresponding character of `strg'. Check the mode
of the matrix.

\hypertarget{b.-5}{%
\subsubsection{b.}\label{b.-5}}

Write a piece of \texttt{R} code to perform the following operation:
Concatenate (without spaces in between) A, C, G, T according to the
values in each row of mat, in such a way that the output is the
following:

{[}1{]} ``GTC'' ``GGA'' ``GTA'' ``TGA''

\hypertarget{c.-5}{%
\subsubsection{c.}\label{c.-5}}

Write a function that concatenates A, C, G, I according to the values in
the row of a general input matrix \texttt{mat} having elements in (0; 1;
2; 3). Check with an example if your function performs well.

\hypertarget{the-elements-of-a-list-object}{%
\subsection{\texorpdfstring{9. The elements of a \texttt{list}
object}{9. The elements of a list object}}\label{the-elements-of-a-list-object}}

Read example 4.2.4 (from p.~90) and try to understand the function
\texttt{findwords} (which is given below). Perform the function step by
step, using the text file \texttt{testconcorda.txt} (from the
\texttt{0\_data} folder). Finally, create an object \texttt{wl}, using
the function \texttt{findwords} with \texttt{testconcorda.txt} as input.
Inspect the class of \texttt{wl} and ask for the component
\texttt{that}.

\begin{Shaded}
\begin{Highlighting}[]
\NormalTok{findwords <-}\StringTok{ }\ControlFlowTok{function}\NormalTok{(tf) \{}
   \CommentTok{# read in the words from the file, into a vector of mode character}
\NormalTok{   txt <-}\StringTok{ }\KeywordTok{scan}\NormalTok{(tf, }\StringTok{""}\NormalTok{)  }
\NormalTok{   wl <-}\StringTok{ }\KeywordTok{list}\NormalTok{()  }
   \ControlFlowTok{for}\NormalTok{ (i }\ControlFlowTok{in} \DecValTok{1}\OperatorTok{:}\KeywordTok{length}\NormalTok{(txt)) \{}
\NormalTok{      wrd <-}\StringTok{ }\NormalTok{txt[i]  }\CommentTok{# i-th word in input file}
\NormalTok{      wl[[wrd]] <-}\StringTok{ }\KeywordTok{c}\NormalTok{(wl[[wrd]], i)  }
\NormalTok{   \}  }
   \KeywordTok{return}\NormalTok{(wl)}
\NormalTok{\}}
\end{Highlighting}
\end{Shaded}

\hypertarget{pmax-and-play-with-logicals}{%
\subsection{\texorpdfstring{10. \texttt{pmax} and play with
logicals}{10. pmax and play with logicals}}\label{pmax-and-play-with-logicals}}

We have the following data frame:

\begin{Shaded}
\begin{Highlighting}[]
\KeywordTok{set.seed}\NormalTok{(}\DecValTok{2009}\NormalTok{) }
\NormalTok{w <-}\StringTok{ }\KeywordTok{runif}\NormalTok{(}\DecValTok{10}\NormalTok{) }
\NormalTok{x <-}\StringTok{ }\KeywordTok{runif}\NormalTok{(}\DecValTok{10}\NormalTok{) }
\NormalTok{y <-}\StringTok{ }\KeywordTok{runif}\NormalTok{(}\DecValTok{10}\NormalTok{) }
\NormalTok{z <-}\StringTok{ }\KeywordTok{runif}\NormalTok{(}\DecValTok{10}\NormalTok{) }
\NormalTok{DF <-}\StringTok{ }\KeywordTok{data.frame}\NormalTok{(}\DataTypeTok{a =}\NormalTok{ w, }\DataTypeTok{b =}\NormalTok{ x, }\DataTypeTok{c =}\NormalTok{ y, }\DataTypeTok{d =}\NormalTok{ z) }
\end{Highlighting}
\end{Shaded}

We define two intervals using the four columns of the data frame, namely
we define the intervals \([\min(a,b), \max(a,b)]\), and
\([\min(c,d), \max(c,d)]\). Add a new logical column in the data frame,
which should be \texttt{TRUE} if the intervals overlap, and
\texttt{FALSE} otherwise. The output should look like:

\begin{Shaded}
\begin{Highlighting}[]
\KeywordTok{head}\NormalTok{(DF)}
\end{Highlighting}
\end{Shaded}

\begin{verbatim}
##             a          b         c          d overlap
## 1 0.197260832 0.03232136 0.5722249 0.05370368    TRUE
## 2 0.696829870 0.25971113 0.5922310 0.10296065    TRUE
## 3 0.607896252 0.57589595 0.8583711 0.90978420   FALSE
## 4 0.009547638 0.82870195 0.4836649 0.82281090    TRUE
## 5 0.429010613 0.67047141 0.4416763 0.74668683    TRUE
## 6 0.076557244 0.57599446 0.1430793 0.49561776    TRUE
\end{verbatim}

Use logical operators and/or if you prefer shorter (and perhaps more
readable) code, use the functions \texttt{pmin} and \texttt{pmax}.

\hypertarget{towards-programming-importing-data-from-excel-using-readxl}{%
\subsection{\texorpdfstring{11. Towards Programming: Importing Data from
Excel using
\texttt{readxl}}{11. Towards Programming: Importing Data from Excel using readxl}}\label{towards-programming-importing-data-from-excel-using-readxl}}

In the upper right pane of RStudio, there is a button called
\texttt{Import\ Dataset}: \FloatBarrier

\begin{figure}[ht]
\centering
\includegraphics{0_image/Import_Dataset.png}
\caption{The option in RStudio you may want to explore.}
\end{figure}

Instead of you using code on the command line in your console
(left-bottom panel), you can explore on how to import your data using
this Graphical User Interface (GUI) in the RStudio editor.

\hypertarget{a.-5}{%
\subsubsection{a.}\label{a.-5}}

Use the above described GUI to produce the code with which you can
import the sheet \texttt{NZA\ 2018} from the
\texttt{Tarievenoverzicht.xlsx} file (see the 0\_data folder). The
\texttt{Tarievenoverzicht.xlsx} contains a pricelist of Health
Insurances and the Nederlandse Zorg Autoriteit (NZA = Dutch Healthcare
Agency) for specialized treatments in the area of Mental Health Care.

\hypertarget{b.-6}{%
\subsubsection{b.}\label{b.-6}}

If all went well with importing the \texttt{NZA\ 2018} data, you could
see that you alos ran \texttt{library(readxl)}. An \texttt{R} package
especially designed for importing data from Excel files.

In this package there is also a function that lists all sheets in the
excel spreadsheet. One of the ways to find this function is to explore
the helpfile of the package, \texttt{help(package\ =\ "readxl")}. Can
you show with the correct \texttt{R} code the names of the sheets in
\texttt{Tarievenoverzicht.xlsx}.

\hypertarget{c.-6}{%
\subsubsection{c.}\label{c.-6}}

Suppose we are only interested in importing the data of the \texttt{NZA}
sheets. Use a combination of the functions \texttt{lapply()},
\texttt{grep}, and the answers of three previous sub-exercises, to
import only the \texttt{NZA} sheets automatically with an
\texttt{lapply} loop. Store the results in a variable \texttt{NZA\_list}

\hypertarget{d.-1}{%
\subsubsection{d.}\label{d.-1}}

Repeat the previous exercise, but now make sure that the class of each
data set in \texttt{NZA\_list} is set equal to \texttt{data.frame} only,
and let each data set consist of five variables:

\begin{Shaded}
\begin{Highlighting}[]
\NormalTok{NZA_varnames <-}\StringTok{ }\KeywordTok{c}\NormalTok{(}\StringTok{"code"}\NormalTok{, }\StringTok{"Productgroup"}\NormalTok{, }\StringTok{"tariff"}\NormalTok{, }\StringTok{"maxtariff"}\NormalTok{)}
\end{Highlighting}
\end{Shaded}

You may use the \texttt{NZA\_varnames}. Note that not all of the data
sets have the fifth \texttt{maxtariff} variable. You'll need a strategy
here to add or create this extra fifth variable yourself, within e.g.~a
customized function for \texttt{lapply()}.

It may be helpfull to first create your own custmized function, e.g.
\texttt{ImportOneSheetNZA}, that performs the needed operations when you
give it the address of \texttt{Tarievenoverzicht} and a sheet name as
input parameters. To stay with the only already used functions in class
and previous exercises, our customized function of th emodel answers
consists of the functions \texttt{read\_excel()}, \texttt{data.frame()}.

\emph{Hint: Instead of using }\texttt{if}* statements or the
\emph{\texttt{ifelse()}} function, we just always created an extra
column, but in the end only selected the columns that where consistent
with* \texttt{NZA\_varnames}\emph{.}

Show that you managed, by neatly printing the new variable names of each
data set.


\end{document}

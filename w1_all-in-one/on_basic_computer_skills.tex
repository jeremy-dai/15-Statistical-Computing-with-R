\documentclass[]{article}
\usepackage{lmodern}
\usepackage{amssymb,amsmath}
\usepackage{ifxetex,ifluatex}
\usepackage{fixltx2e} % provides \textsubscript
\ifnum 0\ifxetex 1\fi\ifluatex 1\fi=0 % if pdftex
  \usepackage[T1]{fontenc}
  \usepackage[utf8]{inputenc}
\else % if luatex or xelatex
  \ifxetex
    \usepackage{mathspec}
  \else
    \usepackage{fontspec}
  \fi
  \defaultfontfeatures{Ligatures=TeX,Scale=MatchLowercase}
\fi
% use upquote if available, for straight quotes in verbatim environments
\IfFileExists{upquote.sty}{\usepackage{upquote}}{}
% use microtype if available
\IfFileExists{microtype.sty}{%
\usepackage{microtype}
\UseMicrotypeSet[protrusion]{basicmath} % disable protrusion for tt fonts
}{}
\usepackage[margin=1in]{geometry}
\usepackage{hyperref}
\hypersetup{unicode=true,
            pdftitle={On some basic computer skills},
            pdfborder={0 0 0},
            breaklinks=true}
\urlstyle{same}  % don't use monospace font for urls
\usepackage{graphicx,grffile}
\makeatletter
\def\maxwidth{\ifdim\Gin@nat@width>\linewidth\linewidth\else\Gin@nat@width\fi}
\def\maxheight{\ifdim\Gin@nat@height>\textheight\textheight\else\Gin@nat@height\fi}
\makeatother
% Scale images if necessary, so that they will not overflow the page
% margins by default, and it is still possible to overwrite the defaults
% using explicit options in \includegraphics[width, height, ...]{}
\setkeys{Gin}{width=\maxwidth,height=\maxheight,keepaspectratio}
\IfFileExists{parskip.sty}{%
\usepackage{parskip}
}{% else
\setlength{\parindent}{0pt}
\setlength{\parskip}{6pt plus 2pt minus 1pt}
}
\setlength{\emergencystretch}{3em}  % prevent overfull lines
\providecommand{\tightlist}{%
  \setlength{\itemsep}{0pt}\setlength{\parskip}{0pt}}
\setcounter{secnumdepth}{0}
% Redefines (sub)paragraphs to behave more like sections
\ifx\paragraph\undefined\else
\let\oldparagraph\paragraph
\renewcommand{\paragraph}[1]{\oldparagraph{#1}\mbox{}}
\fi
\ifx\subparagraph\undefined\else
\let\oldsubparagraph\subparagraph
\renewcommand{\subparagraph}[1]{\oldsubparagraph{#1}\mbox{}}
\fi

%%% Use protect on footnotes to avoid problems with footnotes in titles
\let\rmarkdownfootnote\footnote%
\def\footnote{\protect\rmarkdownfootnote}

%%% Change title format to be more compact
\usepackage{titling}

% Create subtitle command for use in maketitle
\newcommand{\subtitle}[1]{
  \posttitle{
    \begin{center}\large#1\end{center}
    }
}

\setlength{\droptitle}{-2em}
  \title{On some basic computer skills}
  \pretitle{\vspace{\droptitle}\centering\huge}
  \posttitle{\par}
  \author{}
  \preauthor{}\postauthor{}
  \date{}
  \predate{}\postdate{}


\begin{document}
\maketitle

If all is well, you've done or will do the \texttt{swirl} module
``Workspace and Files''. In this module you are asked to set your
workspace (\texttt{getwd()}) and to interact with files on your computer
(e.g. \texttt{list.files()} and \texttt{dir.create()}). A prerequisite
for doing this are some basic computer skills. One of these is
understanding the basic folder structure on your computer. During the
second week we'll use this knowledge to read data from our computer into
\texttt{R}. Another is that most of these data will come in a standard
that is called the ASCII standard, and comes for example in the form of
a \texttt{.txt} file.

If you are (yet) unfamiliar with either the folder structure, or what a
\texttt{.txt} file is (or a \texttt{.csv} file for that matter), or how
to open a \texttt{.txt} file, no problem! However, do familiarize
yourself with basic file extentsions and folder structures. You could
ask a colleague for help, or look at some online resources to educate
yourself. Some example resources:

\begin{itemize}
\tightlist
\item
  \url{https://en.wikipedia.org/wiki/Directory_(computing)}
\item
  \url{https://en.wikipedia.org/wiki/ASCII}
\item
  \url{http://file.org/extension/txt}
\item
  \url{http://www.wikihow.com/Change-Directories-in-Command-Prompt}
\item
  \url{http://mac.appstorm.net/how-to/utilities-how-to/how-to-use-terminal-the-basics/}
\item
  Search yourself (e.g.~YouTube)
\end{itemize}

The possibilities are endless!


\end{document}

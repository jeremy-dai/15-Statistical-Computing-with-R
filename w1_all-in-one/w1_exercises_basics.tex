\documentclass[]{article}
\usepackage{lmodern}
\usepackage{amssymb,amsmath}
\usepackage{ifxetex,ifluatex}
\usepackage{fixltx2e} % provides \textsubscript
\ifnum 0\ifxetex 1\fi\ifluatex 1\fi=0 % if pdftex
  \usepackage[T1]{fontenc}
  \usepackage[utf8]{inputenc}
\else % if luatex or xelatex
  \ifxetex
    \usepackage{mathspec}
  \else
    \usepackage{fontspec}
  \fi
  \defaultfontfeatures{Ligatures=TeX,Scale=MatchLowercase}
\fi
% use upquote if available, for straight quotes in verbatim environments
\IfFileExists{upquote.sty}{\usepackage{upquote}}{}
% use microtype if available
\IfFileExists{microtype.sty}{%
\usepackage{microtype}
\UseMicrotypeSet[protrusion]{basicmath} % disable protrusion for tt fonts
}{}
\usepackage[margin=1in]{geometry}
\usepackage{hyperref}
\hypersetup{unicode=true,
            pdftitle={SCR week 1: exercises},
            pdfborder={0 0 0},
            breaklinks=true}
\urlstyle{same}  % don't use monospace font for urls
\usepackage{color}
\usepackage{fancyvrb}
\newcommand{\VerbBar}{|}
\newcommand{\VERB}{\Verb[commandchars=\\\{\}]}
\DefineVerbatimEnvironment{Highlighting}{Verbatim}{commandchars=\\\{\}}
% Add ',fontsize=\small' for more characters per line
\usepackage{framed}
\definecolor{shadecolor}{RGB}{248,248,248}
\newenvironment{Shaded}{\begin{snugshade}}{\end{snugshade}}
\newcommand{\AlertTok}[1]{\textcolor[rgb]{0.94,0.16,0.16}{#1}}
\newcommand{\AnnotationTok}[1]{\textcolor[rgb]{0.56,0.35,0.01}{\textbf{\textit{#1}}}}
\newcommand{\AttributeTok}[1]{\textcolor[rgb]{0.77,0.63,0.00}{#1}}
\newcommand{\BaseNTok}[1]{\textcolor[rgb]{0.00,0.00,0.81}{#1}}
\newcommand{\BuiltInTok}[1]{#1}
\newcommand{\CharTok}[1]{\textcolor[rgb]{0.31,0.60,0.02}{#1}}
\newcommand{\CommentTok}[1]{\textcolor[rgb]{0.56,0.35,0.01}{\textit{#1}}}
\newcommand{\CommentVarTok}[1]{\textcolor[rgb]{0.56,0.35,0.01}{\textbf{\textit{#1}}}}
\newcommand{\ConstantTok}[1]{\textcolor[rgb]{0.00,0.00,0.00}{#1}}
\newcommand{\ControlFlowTok}[1]{\textcolor[rgb]{0.13,0.29,0.53}{\textbf{#1}}}
\newcommand{\DataTypeTok}[1]{\textcolor[rgb]{0.13,0.29,0.53}{#1}}
\newcommand{\DecValTok}[1]{\textcolor[rgb]{0.00,0.00,0.81}{#1}}
\newcommand{\DocumentationTok}[1]{\textcolor[rgb]{0.56,0.35,0.01}{\textbf{\textit{#1}}}}
\newcommand{\ErrorTok}[1]{\textcolor[rgb]{0.64,0.00,0.00}{\textbf{#1}}}
\newcommand{\ExtensionTok}[1]{#1}
\newcommand{\FloatTok}[1]{\textcolor[rgb]{0.00,0.00,0.81}{#1}}
\newcommand{\FunctionTok}[1]{\textcolor[rgb]{0.00,0.00,0.00}{#1}}
\newcommand{\ImportTok}[1]{#1}
\newcommand{\InformationTok}[1]{\textcolor[rgb]{0.56,0.35,0.01}{\textbf{\textit{#1}}}}
\newcommand{\KeywordTok}[1]{\textcolor[rgb]{0.13,0.29,0.53}{\textbf{#1}}}
\newcommand{\NormalTok}[1]{#1}
\newcommand{\OperatorTok}[1]{\textcolor[rgb]{0.81,0.36,0.00}{\textbf{#1}}}
\newcommand{\OtherTok}[1]{\textcolor[rgb]{0.56,0.35,0.01}{#1}}
\newcommand{\PreprocessorTok}[1]{\textcolor[rgb]{0.56,0.35,0.01}{\textit{#1}}}
\newcommand{\RegionMarkerTok}[1]{#1}
\newcommand{\SpecialCharTok}[1]{\textcolor[rgb]{0.00,0.00,0.00}{#1}}
\newcommand{\SpecialStringTok}[1]{\textcolor[rgb]{0.31,0.60,0.02}{#1}}
\newcommand{\StringTok}[1]{\textcolor[rgb]{0.31,0.60,0.02}{#1}}
\newcommand{\VariableTok}[1]{\textcolor[rgb]{0.00,0.00,0.00}{#1}}
\newcommand{\VerbatimStringTok}[1]{\textcolor[rgb]{0.31,0.60,0.02}{#1}}
\newcommand{\WarningTok}[1]{\textcolor[rgb]{0.56,0.35,0.01}{\textbf{\textit{#1}}}}
\usepackage{graphicx,grffile}
\makeatletter
\def\maxwidth{\ifdim\Gin@nat@width>\linewidth\linewidth\else\Gin@nat@width\fi}
\def\maxheight{\ifdim\Gin@nat@height>\textheight\textheight\else\Gin@nat@height\fi}
\makeatother
% Scale images if necessary, so that they will not overflow the page
% margins by default, and it is still possible to overwrite the defaults
% using explicit options in \includegraphics[width, height, ...]{}
\setkeys{Gin}{width=\maxwidth,height=\maxheight,keepaspectratio}
\IfFileExists{parskip.sty}{%
\usepackage{parskip}
}{% else
\setlength{\parindent}{0pt}
\setlength{\parskip}{6pt plus 2pt minus 1pt}
}
\setlength{\emergencystretch}{3em}  % prevent overfull lines
\providecommand{\tightlist}{%
  \setlength{\itemsep}{0pt}\setlength{\parskip}{0pt}}
\setcounter{secnumdepth}{0}
% Redefines (sub)paragraphs to behave more like sections
\ifx\paragraph\undefined\else
\let\oldparagraph\paragraph
\renewcommand{\paragraph}[1]{\oldparagraph{#1}\mbox{}}
\fi
\ifx\subparagraph\undefined\else
\let\oldsubparagraph\subparagraph
\renewcommand{\subparagraph}[1]{\oldsubparagraph{#1}\mbox{}}
\fi

%%% Use protect on footnotes to avoid problems with footnotes in titles
\let\rmarkdownfootnote\footnote%
\def\footnote{\protect\rmarkdownfootnote}

%%% Change title format to be more compact
\usepackage{titling}

% Create subtitle command for use in maketitle
\providecommand{\subtitle}[1]{
  \posttitle{
    \begin{center}\large#1\end{center}
    }
}

\setlength{\droptitle}{-2em}

  \title{SCR week 1: exercises}
    \pretitle{\vspace{\droptitle}\centering\huge}
  \posttitle{\par}
    \author{}
    \preauthor{}\postauthor{}
    \date{}
    \predate{}\postdate{}
  

\begin{document}
\maketitle

\hypertarget{exercises-part-1-objects-and-types-using-functions}{%
\section{Exercises part 1: Objects and types, using
functions}\label{exercises-part-1-objects-and-types-using-functions}}

\hypertarget{working-with-the-work-space}{%
\subsection{1.1 Working with the work
space}\label{working-with-the-work-space}}

\texttt{R}`s workspace is a nonphysical 'environment' that contains
(remembers) the variables that we construct in our \texttt{R} commands.
This workspace seems empty as we start \texttt{R} up. In this exercise
we will look at some aspects of \texttt{R}'s workspace and scripts.

\hypertarget{a.}{%
\subsubsection{a.}\label{a.}}

Create two variables in \texttt{R}'s console, called \texttt{T} and
\texttt{Y}. Assign the values \texttt{5} and \texttt{20} respectively.

\hypertarget{b.}{%
\subsubsection{b.}\label{b.}}

Write a script that multiplies the values of \texttt{T} and \texttt{Y}.

\hypertarget{c.}{%
\subsubsection{c.}\label{c.}}

Save the script file, with the name \texttt{Separated.R}.

\hypertarget{d.}{%
\subsubsection{d.}\label{d.}}

Close Rstudio, and start it up again. Make sure the scriptfile
\texttt{Separated.R} is loaded into Rstudio (which it will be by default
if you did not explicitly close the script before you exited Rstudio,
and don't save the workspace).

\hypertarget{e.}{%
\subsubsection{e.}\label{e.}}

Use the script to run the multiplication between \texttt{T} and
\texttt{Y} again. Does it work? Why not?

If done correctly, you will notice that \texttt{Y} not longer exists,
and that \texttt{T}, after you restart \texttt{R}, will be synonymous
again with \texttt{TRUE}. Like a few other objects, \texttt{T} is put in
the workspace of \texttt{R}, automatically, but can as we've witnessed
be overwritten. In general it is not a good idea to overwrite existing
object names, such as \texttt{T} or \texttt{F}: if you combine your code
with somebody else's they might have use \texttt{T} instead of
\texttt{TRUE} in their code to check if something is true or not. If you
overwrite \texttt{T}, their code will not work anymore. Similarly: it is
better to use \texttt{TRUE} than \texttt{T}.

\hypertarget{coercion}{%
\subsection{1.2 Coercion}\label{coercion}}

We've seen three modes (or data types): numeric, character and logical.
We've also seen that \texttt{R} will sometimes automatically convert one
type, into another, if it thinks that's what you want it to do. For
example, if we multiply \texttt{TRUE} by 10, the answer is 10. This is
called \emph{implicit coercion}: \texttt{TRUE} is coerced, or forced, to
be interpreted as a \texttt{1}.

\hypertarget{a.-1}{%
\subsubsection{a.}\label{a.-1}}

Try multiplying some numbers with \texttt{TRUE} and \texttt{FALSE}
yourself.

\hypertarget{b.-1}{%
\subsubsection{b.}\label{b.-1}}

Take the following character values and assign them to some objects
(give some sensible names yourself): \texttt{"The\ number\ two"},
\texttt{"2"} and \texttt{"two"}. Use the function \texttt{mode} to check
if the data type of entries is \texttt{character}, \texttt{logical} or
\texttt{numeric}.

\hypertarget{c.-1}{%
\subsubsection{c.}\label{c.-1}}

Try multiplying the objects you've created by 10. Do you get a warning,
or worse, an error?

\hypertarget{d.-1}{%
\subsubsection{d.}\label{d.-1}}

Unfortunately, this does not work (\texttt{R} gives an error). In this
case \texttt{R} does not automatically coerce the 3 objects to numbers.
We can force \texttt{R} to try to coerce the objects to a numeric one
using the function \texttt{as.numeric}. Apply \texttt{as.numeric} to the
objects you've created. Do you get an error, or a warning?

\hypertarget{e.-1}{%
\subsubsection{e.}\label{e.-1}}

We got a warning because \texttt{R} does not know how to convert
\texttt{"The\ number\ two"} or \texttt{"two"} to a number: instead it
turned those objects to \texttt{NA} which stands for
\texttt{Not\ Available} and can be considered a `missing' value.
Amazingly however, \texttt{R} does know how to convert \texttt{"2"}, to
\texttt{2}! Try it the other way around by converting a few numbers to
characters, by using the \texttt{as.character} function. Does this
produce any \texttt{NA}'s?

\hypertarget{f.}{%
\subsubsection{f.}\label{f.}}

Take the values \texttt{-3}, \texttt{-1}, \texttt{0}, \texttt{1} and
\texttt{1000} and coerce each to the logical type, by using the function
\texttt{as.logical}. What is the result?

\hypertarget{g.}{%
\subsubsection{g.}\label{g.}}

What do you think will happen in the following call:
\texttt{as.character(TRUE)} And in: \texttt{as.logical("TRUE")}, or
\texttt{as.logical("completely\ false")}?

\newpage

\hypertarget{exercises-part-2-vectors-and-functions}{%
\section{Exercises part 2: vectors and
functions}\label{exercises-part-2-vectors-and-functions}}

\hypertarget{operations-on-vectors}{%
\subsection{2.1 Operations on vectors}\label{operations-on-vectors}}

\hypertarget{a.-2}{%
\subsubsection{a.}\label{a.-2}}

Create a vector containing the values 0.2, 0.4, 0.6, \ldots 1.8, and
2.0. The vector should consist of 10 elements.

\hypertarget{b.-2}{%
\subsubsection{b.}\label{b.-2}}

Go to the vocabulary that's put online by Hadley Wickham:
\url{http://adv-r.had.co.nz/Vocabulary.html}. Look at the operators
under the \textbf{basic math} header. Try a few of the following
operators:

*, +, -, /, \^{}, \%\%, \%/\% abs, sign acos, asin, atan, atan2 sin,
cos, tan ceiling, floor, round, trunc, signif exp, log, log10, log2,
sqrt

On which element(s) of the vector \texttt{my\_vec} does the function
operates?

\hypertarget{c.-2}{%
\subsubsection{c.}\label{c.-2}}

The vocabulary also lists the following operators:

max, min, prod, sum cummax, cummin, cumprod, cumsum, diff pmax, pmin
range mean, median, cor, sd, var rle

Try a few of these. Look at the helpfiles of a function (using e.g.
\texttt{?max}) if you don't know what the function does. What is the big
difference between these operators and the ones you tried in
\textbf{b.}?

\hypertarget{creating-a-typical-function}{%
\subsection{2.2 Creating a typical
function}\label{creating-a-typical-function}}

You've seen how to create a function:

\begin{Shaded}
\begin{Highlighting}[]
\NormalTok{FunctionName <-}\StringTok{ }\ControlFlowTok{function}\NormalTok{(argument)\{}
  
  \CommentTok{# do stuff }
  
  \KeywordTok{return}\NormalTok{(return_value)}
\NormalTok{\}}
\end{Highlighting}
\end{Shaded}

In this exercise we will walk through the typical process one might go
through in create a function and look a the concept of functions and
\emph{scoping}.

\hypertarget{a.-3}{%
\subsubsection{a.}\label{a.-3}}

Create a vector using the following code:
\texttt{my\_vector\ \textless{}-\ c(4,\ 70,\ 19,\ 21,\ 77,\ 82,\ 75,\ 33,\ 90,\ 34,\ 6,\ 27,\ 63,\ 25,\ 39,\ 83,\ 42,\ 60,\ 17,\ 10)}.

\hypertarget{b.-3}{%
\subsubsection{b.}\label{b.-3}}

Find the minimal value of the vector using the function \texttt{min()}.

\hypertarget{c.-3}{%
\subsubsection{c.}\label{c.-3}}

Find the maximal value of the vector using the function \texttt{max()}.

\hypertarget{d.-2}{%
\subsubsection{d.}\label{d.-2}}

Add the maximum and minimum value together, divide by 2 and substract
that value from the \texttt{mean()} of the vector. Which is bigger?

\hypertarget{e.-2}{%
\subsubsection{e.}\label{e.-2}}

Write code to have \texttt{R} tell you whether it is \texttt{TRUE} or
\texttt{FALSE} that the mean is bigger than the `halfway' value of the
range of our vector.

\hypertarget{f.-1}{%
\subsubsection{f.}\label{f.-1}}

Write a function called \texttt{IsMeanBiggerThanHalfway} that takes as
argument a vector and returns \texttt{TRUE} or \texttt{FALSE} depending
on whether the mean is bigger that the halfway value of the range of the
vector that is entered as argument.

\hypertarget{creating-a-function-with-multiple-arguments}{%
\subsection{2.3 Creating a function with multiple
arguments}\label{creating-a-function-with-multiple-arguments}}

You'll often see functions in \texttt{R} that can take multiple
arguments. The result of the function (usually the return value) will
depend on \emph{both} arguments. \texttt{sd} is a function that takes
two arguments. Without looking exactly how \texttt{sd} works, look at
the helpfile of the \texttt{sd} en try to figure out from the syntax
presented in the helpfile (under \textbf{Usage}) how to create a
function that takes 3 arguments.

Write a function (choose an active name yourself) that takes three
arguments. Let the function return the product of the three values.

\hypertarget{some-vector-exercises-again-slightly-more-difficult-for-now}{%
\subsection{2.4. Some Vector Exercises Again (slightly more difficult,
for
now)}\label{some-vector-exercises-again-slightly-more-difficult-for-now}}

While using \texttt{rep()}, \texttt{seq()} and/or arithemtic thinking,
generate the following sequences:

\begin{itemize}
\item[(a)] $10, 8, 6, 4, 6, 8, 10$
\item[(b)] $60, 56, 52, \ldots, 12, 8$.
\item[(c)] $1, 2, 4, 8, \ldots, 512$
\item[(d)] $0, 1, 2, 0, \ldots, 2, 0, 1, 2$ (with each entry appearing six times)
\item[(e)] $1, 2, 2, 3, 3, 3, 4, 4, 4$.
\item[(f)] $1, 2, 5, 10, 20, 50, 100, \ldots, 5 x 10^3$ (use vector recycling!)
\end{itemize}

\hypertarget{dont-stop-til-you-get-enough-more-difficult-for-now}{%
\subsection{2.5. Don't Stop 'til You Get Enough (more difficult, for
now)}\label{dont-stop-til-you-get-enough-more-difficult-for-now}}

While using \texttt{cos()}, exp(), \texttt{\%\%}, \texttt{:}, and/or
arithemtic thinking, generate the following sequences:

\begin{itemize}
\item[(a)] $cos\left( \frac{\pi n}{3}\right), \text{ for } n = \{0, \ldots, 10\}.$
\item[(b)] $1, 9, 98,997, \ldots, 999994.$
\item[(c)] $e^n - 3n, \text{ for } n = \{0, \ldots, 10\}.$
\item[(d)] $3n \mbox{ mod } 7, \text{ for } n = \{0, \ldots, 10\}.$
\item[(e)] Let
$$ \tilde{\pi}_n = 4 \sum^n_{i = 1} \frac{(-1)^{i + 1}}{2i - 1} = 4 - \frac{4}{3} + \frac{4}{5} - \frac{4}{7} + \ldots + 4\frac{(-1)^{n + 1}}{2n - 1}.$$ Create a function `ApproxPi` that outputs $\pi_n$ when you give it $n$. You may want to use a `sum()` function on a vector, and you could use vector recycling. Evaluate $\tilde{\pi}_{100}$, $\tilde{\pi}_{1000}$, and $\tilde{\pi}_{10000}$, do you notice anything?
\end{itemize}

\newpage

\hypertarget{exercises-part-3-conditions-and-if-indexing-and-filtering}{%
\section{Exercises part 3: Conditions and if, indexing and
filtering}\label{exercises-part-3-conditions-and-if-indexing-and-filtering}}

\hypertarget{an-absolute-function}{%
\subsection{3.1. An absolute function}\label{an-absolute-function}}

In this exercise we will create a function that returns the absolute
difference between two values. Say we wish to find the absolute
difference of the expression \(4 - 10\) is equal to \(6\). We could use
the following statement to find the absolute difference between
\texttt{x} and \texttt{y} regardless of which is bigger:

\begin{Shaded}
\begin{Highlighting}[]
\KeywordTok{abs}\NormalTok{(x}\OperatorTok{-}\NormalTok{y)}
\end{Highlighting}
\end{Shaded}

Regardless of whether the result is positive or negative, that
\texttt{abs} function will make it a positive number. We'll implement a
function ourselves using conditions and an \texttt{if} statement.

\hypertarget{a.-4}{%
\subsubsection{a.}\label{a.-4}}

Write a line of code that substracts \texttt{y} from \texttt{x} and
saves it as a new value \texttt{z}. To test your code you will need to
assign some values to \texttt{x} and \texttt{y}.

\hypertarget{b.-4}{%
\subsubsection{b.}\label{b.-4}}

Write a condition (or logical expression) that checks whether \texttt{z}
is negative.

\hypertarget{c.-4}{%
\subsubsection{c.}\label{c.-4}}

Write an \texttt{if} statement that uses the condition you've written
above to check whether \texttt{z} is negative, and if so, multiplies
\texttt{z} by \(-1\) to make it positive.

\hypertarget{d.-3}{%
\subsubsection{d.}\label{d.-3}}

Put the above code into a function called \texttt{AbsoluteDifference}
and test the code with the following value pairs:

\begin{itemize}
\tightlist
\item
  \(x=10\), \(y=4\)
\item
  \(x=4\), \(y=10\)
\item
  \(x=4\), \(y=-10\)
\end{itemize}

Did your function test correctly?

\hypertarget{our-first-filter}{%
\subsection{3.2. Our first filter}\label{our-first-filter}}

We've seen during class that we can index in a variety of ways: with
positions, with negative positions, with names and with logicals. In
this exercise we'll be using logicals to create a filter.

\hypertarget{a.-5}{%
\subsubsection{a.}\label{a.-5}}

Create a vector, called short\_alphabet, containing the first 10 (no
capital) letters of the alphabet.

\hypertarget{b.-5}{%
\subsubsection{b.}\label{b.-5}}

Use brackets (\texttt{{[}{]}}) to select the 7th letter.

\hypertarget{c.-5}{%
\subsubsection{c.}\label{c.-5}}

Besides giving the position (or positions!) of the elements you want to
access you can also tell \texttt{R} which elements you \textbf{do} and
which elements you \textbf{don't} want to select, by telling \texttt{R}
for each position whether it is \texttt{TRUE} or \texttt{FALSE} that you
want to select each element.

For example, we can select the second and fourth element of the vector
\texttt{c(1,\ 2,\ 3,\ 4,\ 5)} in the following way:

\begin{Shaded}
\begin{Highlighting}[]
\NormalTok{a <-}\StringTok{ }\KeywordTok{c}\NormalTok{(}\DecValTok{1}\NormalTok{, }\DecValTok{2}\NormalTok{, }\DecValTok{3}\NormalTok{, }\DecValTok{4}\NormalTok{, }\DecValTok{5}\NormalTok{)}
\NormalTok{a[}\KeywordTok{c}\NormalTok{(}\OtherTok{FALSE}\NormalTok{, }\OtherTok{TRUE}\NormalTok{, }\OtherTok{FALSE}\NormalTok{, }\OtherTok{TRUE}\NormalTok{, }\OtherTok{FALSE}\NormalTok{)]}
\end{Highlighting}
\end{Shaded}

\begin{verbatim}
## [1] 2 4
\end{verbatim}

Try this yourself by creating a vector, containing only \texttt{TRUE}
and \texttt{FALSE}, that you can use to select the 7th letter from the
\texttt{short\_alphabet} object.

\hypertarget{d.-4}{%
\subsubsection{d.}\label{d.-4}}

Instead of manually typing a vector of \texttt{TRUE} and \texttt{FALSE}
we can use \texttt{R} and its vectorized functions to create one for us.
In the previous exercise we've learned that \texttt{R} can evaluate a
logical expression to \texttt{TRUE} or \texttt{FALSE}. It can do this in
a vectorized manner. An example of a vectorized function is
multiplication: if you have a vector of numbers, you can multiply the
vector object with a constant, to multiply \emph{all elements} in the
vector with that number. For example:

\begin{Shaded}
\begin{Highlighting}[]
\NormalTok{a <-}\StringTok{ }\KeywordTok{c}\NormalTok{(}\DecValTok{1}\NormalTok{, }\DecValTok{2}\NormalTok{, }\DecValTok{3}\NormalTok{, }\DecValTok{4}\NormalTok{, }\DecValTok{5}\NormalTok{)}
\NormalTok{a }\OperatorTok{*}\StringTok{ }\DecValTok{2}
\end{Highlighting}
\end{Shaded}

\begin{verbatim}
## [1]  2  4  6  8 10
\end{verbatim}

We can do a similar thing with a condition that checks for equality:

\begin{Shaded}
\begin{Highlighting}[]
\NormalTok{short_alphabet }\OperatorTok{==}\StringTok{ "a"}
\end{Highlighting}
\end{Shaded}

Use this to create a vector that has a \texttt{TRUE} only in the 7th
position. Save this to an object called \texttt{seventh\_letter}.

\hypertarget{e.-3}{%
\subsubsection{e.}\label{e.-3}}

Create another vector with only \texttt{TRUE} and \texttt{FALSE}, but
one with just a \texttt{TRUE} in the position of the letter `c'. Save it
to an object called \texttt{third\_letter}.

\hypertarget{f.-2}{%
\subsubsection{f.}\label{f.-2}}

Our first `filter' will be created by combining the two vectors
containing only \texttt{TRUE} and \texttt{FALSE}. We won't skip ahead
just yet and talk about more advanced parts of \emph{control}
statements. Instead, we will use a trick. We've already seen that we can
use \texttt{TRUE} and \texttt{FALSE} for calculation: if forced to be
read as a number,\texttt{TRUE} is equal to 1, and \texttt{FALSE} is
equal to 0. Thus, if we add, pairwise, the elements of both vectors
everything that was \texttt{FALSE} in both cases will be \texttt{0}, and
everything that was \texttt{TRUE} in either or both will be \texttt{1}
or \texttt{2}. If we coerce anything but \texttt{0} to a logical,
\texttt{R} will make it \texttt{TRUE}. Thus we can add one vector to the
other, coerce it to a logical vector and use it to subset (or index) our
vector of numbers. Do this now for the letters \texttt{c} and
\texttt{g}.

\hypertarget{subsetting}{%
\subsubsection{3.3 Subsetting}\label{subsetting}}

Create a vector \texttt{x} of normal random variables as follows:

\begin{Shaded}
\begin{Highlighting}[]
\KeywordTok{set.seed}\NormalTok{(}\DecValTok{123}\NormalTok{)}
\NormalTok{x <-}\StringTok{ }\KeywordTok{rnorm}\NormalTok{(}\DecValTok{1000}\NormalTok{)}
\end{Highlighting}
\end{Shaded}

The \texttt{set.seed()} fixes the random number generator so that we all
obtain the same \texttt{x}; changing the argument \texttt{123} to
something else will give different results. This is useful for
replication.

\begin{itemize}
\item[(a)] show the first 10 elements from the vector `x`
\item[(b)] show each 100th element of `x`
\item[(c)] show how many of the elements are $> 1$ or $< -1$ 
\item[(d)] show the proportion of elements higher than 1.645 
\end{itemize}


\end{document}

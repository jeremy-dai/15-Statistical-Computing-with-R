\documentclass[]{article}
\usepackage{lmodern}
\usepackage{amssymb,amsmath}
\usepackage{ifxetex,ifluatex}
\usepackage{fixltx2e} % provides \textsubscript
\ifnum 0\ifxetex 1\fi\ifluatex 1\fi=0 % if pdftex
  \usepackage[T1]{fontenc}
  \usepackage[utf8]{inputenc}
\else % if luatex or xelatex
  \ifxetex
    \usepackage{mathspec}
  \else
    \usepackage{fontspec}
  \fi
  \defaultfontfeatures{Ligatures=TeX,Scale=MatchLowercase}
\fi
% use upquote if available, for straight quotes in verbatim environments
\IfFileExists{upquote.sty}{\usepackage{upquote}}{}
% use microtype if available
\IfFileExists{microtype.sty}{%
\usepackage{microtype}
\UseMicrotypeSet[protrusion]{basicmath} % disable protrusion for tt fonts
}{}
\usepackage[margin=1in]{geometry}
\usepackage{hyperref}
\hypersetup{unicode=true,
            pdftitle={A Colorful Christmas:},
            pdfauthor={Yizhen (Jeremy) Dai / S2395479},
            pdfborder={0 0 0},
            breaklinks=true}
\urlstyle{same}  % don't use monospace font for urls
\usepackage{color}
\usepackage{fancyvrb}
\newcommand{\VerbBar}{|}
\newcommand{\VERB}{\Verb[commandchars=\\\{\}]}
\DefineVerbatimEnvironment{Highlighting}{Verbatim}{commandchars=\\\{\}}
% Add ',fontsize=\small' for more characters per line
\usepackage{framed}
\definecolor{shadecolor}{RGB}{248,248,248}
\newenvironment{Shaded}{\begin{snugshade}}{\end{snugshade}}
\newcommand{\AlertTok}[1]{\textcolor[rgb]{0.94,0.16,0.16}{#1}}
\newcommand{\AnnotationTok}[1]{\textcolor[rgb]{0.56,0.35,0.01}{\textbf{\textit{#1}}}}
\newcommand{\AttributeTok}[1]{\textcolor[rgb]{0.77,0.63,0.00}{#1}}
\newcommand{\BaseNTok}[1]{\textcolor[rgb]{0.00,0.00,0.81}{#1}}
\newcommand{\BuiltInTok}[1]{#1}
\newcommand{\CharTok}[1]{\textcolor[rgb]{0.31,0.60,0.02}{#1}}
\newcommand{\CommentTok}[1]{\textcolor[rgb]{0.56,0.35,0.01}{\textit{#1}}}
\newcommand{\CommentVarTok}[1]{\textcolor[rgb]{0.56,0.35,0.01}{\textbf{\textit{#1}}}}
\newcommand{\ConstantTok}[1]{\textcolor[rgb]{0.00,0.00,0.00}{#1}}
\newcommand{\ControlFlowTok}[1]{\textcolor[rgb]{0.13,0.29,0.53}{\textbf{#1}}}
\newcommand{\DataTypeTok}[1]{\textcolor[rgb]{0.13,0.29,0.53}{#1}}
\newcommand{\DecValTok}[1]{\textcolor[rgb]{0.00,0.00,0.81}{#1}}
\newcommand{\DocumentationTok}[1]{\textcolor[rgb]{0.56,0.35,0.01}{\textbf{\textit{#1}}}}
\newcommand{\ErrorTok}[1]{\textcolor[rgb]{0.64,0.00,0.00}{\textbf{#1}}}
\newcommand{\ExtensionTok}[1]{#1}
\newcommand{\FloatTok}[1]{\textcolor[rgb]{0.00,0.00,0.81}{#1}}
\newcommand{\FunctionTok}[1]{\textcolor[rgb]{0.00,0.00,0.00}{#1}}
\newcommand{\ImportTok}[1]{#1}
\newcommand{\InformationTok}[1]{\textcolor[rgb]{0.56,0.35,0.01}{\textbf{\textit{#1}}}}
\newcommand{\KeywordTok}[1]{\textcolor[rgb]{0.13,0.29,0.53}{\textbf{#1}}}
\newcommand{\NormalTok}[1]{#1}
\newcommand{\OperatorTok}[1]{\textcolor[rgb]{0.81,0.36,0.00}{\textbf{#1}}}
\newcommand{\OtherTok}[1]{\textcolor[rgb]{0.56,0.35,0.01}{#1}}
\newcommand{\PreprocessorTok}[1]{\textcolor[rgb]{0.56,0.35,0.01}{\textit{#1}}}
\newcommand{\RegionMarkerTok}[1]{#1}
\newcommand{\SpecialCharTok}[1]{\textcolor[rgb]{0.00,0.00,0.00}{#1}}
\newcommand{\SpecialStringTok}[1]{\textcolor[rgb]{0.31,0.60,0.02}{#1}}
\newcommand{\StringTok}[1]{\textcolor[rgb]{0.31,0.60,0.02}{#1}}
\newcommand{\VariableTok}[1]{\textcolor[rgb]{0.00,0.00,0.00}{#1}}
\newcommand{\VerbatimStringTok}[1]{\textcolor[rgb]{0.31,0.60,0.02}{#1}}
\newcommand{\WarningTok}[1]{\textcolor[rgb]{0.56,0.35,0.01}{\textbf{\textit{#1}}}}
\usepackage{graphicx,grffile}
\makeatletter
\def\maxwidth{\ifdim\Gin@nat@width>\linewidth\linewidth\else\Gin@nat@width\fi}
\def\maxheight{\ifdim\Gin@nat@height>\textheight\textheight\else\Gin@nat@height\fi}
\makeatother
% Scale images if necessary, so that they will not overflow the page
% margins by default, and it is still possible to overwrite the defaults
% using explicit options in \includegraphics[width, height, ...]{}
\setkeys{Gin}{width=\maxwidth,height=\maxheight,keepaspectratio}
\IfFileExists{parskip.sty}{%
\usepackage{parskip}
}{% else
\setlength{\parindent}{0pt}
\setlength{\parskip}{6pt plus 2pt minus 1pt}
}
\setlength{\emergencystretch}{3em}  % prevent overfull lines
\providecommand{\tightlist}{%
  \setlength{\itemsep}{0pt}\setlength{\parskip}{0pt}}
\setcounter{secnumdepth}{0}
% Redefines (sub)paragraphs to behave more like sections
\ifx\paragraph\undefined\else
\let\oldparagraph\paragraph
\renewcommand{\paragraph}[1]{\oldparagraph{#1}\mbox{}}
\fi
\ifx\subparagraph\undefined\else
\let\oldsubparagraph\subparagraph
\renewcommand{\subparagraph}[1]{\oldsubparagraph{#1}\mbox{}}
\fi

%%% Use protect on footnotes to avoid problems with footnotes in titles
\let\rmarkdownfootnote\footnote%
\def\footnote{\protect\rmarkdownfootnote}

%%% Change title format to be more compact
\usepackage{titling}

% Create subtitle command for use in maketitle
\providecommand{\subtitle}[1]{
  \posttitle{
    \begin{center}\large#1\end{center}
    }
}

\setlength{\droptitle}{-2em}

  \title{A Colorful Christmas:}
    \pretitle{\vspace{\droptitle}\centering\huge}
  \posttitle{\par}
  \subtitle{about Stats and Floyd-Steinberg Dithering}
  \author{Yizhen (Jeremy) Dai / S2395479}
    \preauthor{\centering\large\emph}
  \postauthor{\par}
      \predate{\centering\large\emph}
  \postdate{\par}
    \date{07 January, 2020}

\usepackage{graphicx}
\usepackage{float}
\usepackage{placeins}
\usepackage{amsmath}
\newcommand{\argmin}{\operatornamewithlimits{argmin}}
\newcommand{\argmax}{\operatornamewithlimits{argmax}}

\begin{document}
\maketitle

\hypertarget{about-colors-a-picture-a-data-set.}{%
\section{1 About Colors: A Picture = A Data
set.}\label{about-colors-a-picture-a-data-set.}}

\hypertarget{explore-your-png-r-object}{%
\subsection{1.1 Explore your PNG R
object}\label{explore-your-png-r-object}}

Load the file.

\begin{Shaded}
\begin{Highlighting}[]
\CommentTok{# file.info("0_img/xmas.png")[, c("size", "mtime")]}
\NormalTok{xmas <-}\StringTok{ }\NormalTok{png}\OperatorTok{::}\KeywordTok{readPNG}\NormalTok{(}\DataTypeTok{source =} \StringTok{'0_img/xmas.png'}\NormalTok{, }\DataTypeTok{native =} \OtherTok{FALSE}\NormalTok{)}
\NormalTok{xmas <-}\StringTok{ }\NormalTok{xmas[,,}\OperatorTok{-}\DecValTok{4}\NormalTok{] }\CommentTok{# first three matrices only}
\end{Highlighting}
\end{Shaded}

Show the Red, Green and Blue values for the pixel in row 106, and column
467:

\begin{Shaded}
\begin{Highlighting}[]
\NormalTok{color <-}\StringTok{ }\NormalTok{xmas[}\DecValTok{106}\NormalTok{,}\DecValTok{467}\NormalTok{,]}
\NormalTok{color}
\end{Highlighting}
\end{Shaded}

\begin{verbatim}
## [1] 0.9215686 0.0000000 0.0000000
\end{verbatim}

It should be dark red. Let's Draw the color:

\begin{Shaded}
\begin{Highlighting}[]
\KeywordTok{plot}\NormalTok{(}\OtherTok{NULL}\NormalTok{, }\DataTypeTok{xlim=}\KeywordTok{c}\NormalTok{(}\DecValTok{0}\NormalTok{, }\DecValTok{1}\NormalTok{), }\DataTypeTok{ylim=}\KeywordTok{c}\NormalTok{(}\DecValTok{0}\NormalTok{, }\DecValTok{1}\NormalTok{),}\DataTypeTok{axes=}\OtherTok{FALSE}\NormalTok{, }\DataTypeTok{frame.plot=}\OtherTok{TRUE}\NormalTok{, }\DataTypeTok{ann =} \OtherTok{FALSE}\NormalTok{)}
\KeywordTok{rect}\NormalTok{(}\DecValTok{0}\NormalTok{, }\DecValTok{0}\NormalTok{, }\DecValTok{1}\NormalTok{, }\DecValTok{1}\NormalTok{, }\DataTypeTok{col =} \KeywordTok{rgb}\NormalTok{(color[}\DecValTok{1}\NormalTok{],color[}\DecValTok{2}\NormalTok{],color[}\DecValTok{3}\NormalTok{]))}
\end{Highlighting}
\end{Shaded}

\begin{center}\includegraphics{S2395479_Dai_files/figure-latex/unnamed-chunk-4-1} \end{center}

\hypertarget{from-an-rgb-array-to-a-data.frame}{%
\subsection{1.2 From an RGB array to a
data.frame}\label{from-an-rgb-array-to-a-data.frame}}

Create the dataframe with first 5 variables: row, column, red, green,
and blue

\begin{Shaded}
\begin{Highlighting}[]
\NormalTok{index <-}\StringTok{ }\KeywordTok{expand.grid}\NormalTok{(}\DecValTok{1}\OperatorTok{:}\KeywordTok{dim}\NormalTok{(xmas)[}\DecValTok{1}\NormalTok{], }\DecValTok{1}\OperatorTok{:}\KeywordTok{dim}\NormalTok{(xmas)[}\DecValTok{2}\NormalTok{]) }\OperatorTok\StringTok{ }\KeywordTok{as.matrix}\NormalTok{()}
\NormalTok{df_xmas <-}\StringTok{ }\KeywordTok{matrix}\NormalTok{(}\KeywordTok{rep}\NormalTok{(}\DecValTok{0}\NormalTok{,}\KeywordTok{dim}\NormalTok{(xmas)[}\DecValTok{1}\NormalTok{] }\OperatorTok{*}\StringTok{ }\KeywordTok{dim}\NormalTok{(xmas)[}\DecValTok{2}\NormalTok{] }\OperatorTok{*}\StringTok{ }\DecValTok{5}\NormalTok{), }\DataTypeTok{ncol =} \DecValTok{5}\NormalTok{) }\CommentTok{#dummy}

\ControlFlowTok{for}\NormalTok{ (i }\ControlFlowTok{in} \DecValTok{1}\OperatorTok{:}\KeywordTok{nrow}\NormalTok{(index))\{}
\NormalTok{  df_xmas[i,}\DecValTok{1}\NormalTok{] =}\StringTok{ }\NormalTok{index[i,}\DecValTok{1}\NormalTok{] }\CommentTok{#row}
\NormalTok{  df_xmas[i,}\DecValTok{2}\NormalTok{] =}\StringTok{ }\NormalTok{index[i,}\DecValTok{2}\NormalTok{] }\CommentTok{#col}
\NormalTok{  color =}\StringTok{ }\NormalTok{xmas[index[i,}\DecValTok{1}\NormalTok{],index[i,}\DecValTok{2}\NormalTok{],]}
\NormalTok{  df_xmas[i,}\DecValTok{3}\NormalTok{] =}\StringTok{ }\NormalTok{color[}\DecValTok{1}\NormalTok{] }\CommentTok{# red}
\NormalTok{  df_xmas[i,}\DecValTok{4}\NormalTok{] =}\StringTok{ }\NormalTok{color[}\DecValTok{2}\NormalTok{] }\CommentTok{#green}
\NormalTok{  df_xmas[i,}\DecValTok{5}\NormalTok{] =}\StringTok{ }\NormalTok{color[}\DecValTok{3}\NormalTok{] }\CommentTok{#blue}
\NormalTok{\} }
\KeywordTok{remove}\NormalTok{(i)}
\end{Highlighting}
\end{Shaded}

Create the 6th variable: rgb\_color

\begin{Shaded}
\begin{Highlighting}[]
\NormalTok{df_xmas <-}\StringTok{ }\KeywordTok{as.data.frame}\NormalTok{(df_xmas)}
\KeywordTok{names}\NormalTok{(df_xmas) <-}\StringTok{ }\KeywordTok{c}\NormalTok{(}\StringTok{'row'}\NormalTok{, }\StringTok{'col'}\NormalTok{, }\StringTok{'red'}\NormalTok{, }\StringTok{'green'}\NormalTok{, }\StringTok{'blue'}\NormalTok{)}

\NormalTok{df_xmas<-}\StringTok{ }\NormalTok{df_xmas }\OperatorTok
\StringTok{  }\KeywordTok{mutate}\NormalTok{(}\DataTypeTok{rgb_color =} \KeywordTok{rgb}\NormalTok{(red,green,blue)) }\CommentTok{#rgb_color column}

\NormalTok{df_xmas}\OperatorTok{$}\NormalTok{rgb_color =}\StringTok{  }\KeywordTok{as.factor}\NormalTok{(df_xmas}\OperatorTok{$}\NormalTok{rgb_color) }\CommentTok{# convert to factor}
\end{Highlighting}
\end{Shaded}

Check:

\begin{Shaded}
\begin{Highlighting}[]
\KeywordTok{all.equal}\NormalTok{(df_xmas, xmas_df)}
\end{Highlighting}
\end{Shaded}

\hypertarget{number-of-unique-colors-in-xmas.png}{%
\subsection{1.3 Number of Unique colors in
xmas.png}\label{number-of-unique-colors-in-xmas.png}}

\begin{Shaded}
\begin{Highlighting}[]
\NormalTok{df_xmas}\OperatorTok{$}\NormalTok{rgb_color }\OperatorTok\StringTok{ }\KeywordTok{levels}\NormalTok{() }\OperatorTok\StringTok{ }\KeywordTok{length}\NormalTok{()}
\end{Highlighting}
\end{Shaded}

\begin{Shaded}
\begin{Highlighting}[]
\NormalTok{df_xmas }\OperatorTok
\StringTok{  }\KeywordTok{group_by}\NormalTok{(red,green,blue) }\OperatorTok
\StringTok{  }\KeywordTok{tally}\NormalTok{() }\OperatorTok\StringTok{ }
\StringTok{  }\KeywordTok{dim}\NormalTok{()}
\end{Highlighting}
\end{Shaded}

Both have 147839 unique values.

\hypertarget{creating-a-raster-to-plot-the-picture-in-r}{%
\subsection{1.4 Creating A Raster To Plot the Picture in
R}\label{creating-a-raster-to-plot-the-picture-in-r}}

\begin{Shaded}
\begin{Highlighting}[]
\NormalTok{raster_rgb <-}\StringTok{ }\NormalTok{df_xmas }\OperatorTok\StringTok{ }
\StringTok{  }\KeywordTok{select}\NormalTok{(row,col,rgb_color) }\OperatorTok\StringTok{ }
\StringTok{  }\KeywordTok{pivot_wider}\NormalTok{(}\DataTypeTok{names_from=}\StringTok{'col'}\NormalTok{,}\DataTypeTok{values_from =} \StringTok{'rgb_color'}\NormalTok{)  }\OperatorTok\StringTok{ }
\StringTok{  }\KeywordTok{subset}\NormalTok{(., }\DataTypeTok{select =} \OperatorTok{-}\KeywordTok{c}\NormalTok{(row) )  }\OperatorTok\StringTok{ }
\StringTok{  }\KeywordTok{as.matrix}\NormalTok{() }\OperatorTok\StringTok{ }
\StringTok{  }\KeywordTok{as.raster}\NormalTok{()}
\end{Highlighting}
\end{Shaded}

\begin{Shaded}
\begin{Highlighting}[]
\NormalTok{my_pic_}\DecValTok{1}\NormalTok{ <-}\StringTok{ }\KeywordTok{plot}\NormalTok{(raster_rgb)}
\end{Highlighting}
\end{Shaded}

\clearpage

\hypertarget{a-further-understanding-of-the-rgb-space}{%
\section{2 A Further Understanding of the RGB
Space}\label{a-further-understanding-of-the-rgb-space}}

\hypertarget{hexadecimal-identifiers-for-the-rgb-colors}{%
\subsection{2.1 Hexadecimal identifiers for the RGB
colors}\label{hexadecimal-identifiers-for-the-rgb-colors}}

\begin{Shaded}
\begin{Highlighting}[]
\NormalTok{hexadecimal <-}\StringTok{ }\KeywordTok{c}\NormalTok{(}\DecValTok{0}\OperatorTok{:}\DecValTok{9}\NormalTok{, LETTERS[}\DecValTok{1}\OperatorTok{:}\DecValTok{6}\NormalTok{])}
\NormalTok{hdm2columns <-}\StringTok{ }\KeywordTok{expand.grid}\NormalTok{(hexadecimal, hexadecimal)}
\NormalTok{channel <-}\StringTok{ }\KeywordTok{paste0}\NormalTok{(hdm2columns[,}\DecValTok{2}\NormalTok{], hdm2columns[,}\DecValTok{1}\NormalTok{], }\DataTypeTok{sep =} \StringTok{""}\NormalTok{)}
\end{Highlighting}
\end{Shaded}

Create the function:

\begin{Shaded}
\begin{Highlighting}[]
\NormalTok{get_rgb <-}\StringTok{ }\ControlFlowTok{function}\NormalTok{(}\DataTypeTok{red =} \FloatTok{0.5}\NormalTok{, }\DataTypeTok{green =} \FloatTok{0.3}\NormalTok{, }\DataTypeTok{blue =} \FloatTok{0.7}\NormalTok{, }\DataTypeTok{maxColorValue =} \FloatTok{1.}\NormalTok{)\{}
  \CommentTok{# floor(x+0.5) for rounding; channel[x+1] for getting hexadecimal}
\NormalTok{  hex<-}\StringTok{ }\NormalTok{(}\KeywordTok{c}\NormalTok{(red,green, blue)}\OperatorTok{*}\NormalTok{(}\DecValTok{255}\OperatorTok{/}\NormalTok{maxColorValue) }\OperatorTok{+}\StringTok{ }\FloatTok{0.5} \OperatorTok{+}\StringTok{ }\DecValTok{1}\NormalTok{)  }\OperatorTok\StringTok{ }
\StringTok{    }\KeywordTok{floor}\NormalTok{(.)   }\OperatorTok
\StringTok{    }\NormalTok{channel[.]  }\OperatorTok\StringTok{ }
\StringTok{    }\KeywordTok{paste}\NormalTok{(., }\DataTypeTok{collapse =} \StringTok{''}\NormalTok{)}
  \KeywordTok{return}\NormalTok{(}\KeywordTok{paste0}\NormalTok{(}\StringTok{'#'}\NormalTok{,hex))}
\NormalTok{\}}
\KeywordTok{get_rgb}\NormalTok{(}\DataTypeTok{maxColorValue =} \FloatTok{1.0}\NormalTok{)}
\KeywordTok{rgb}\NormalTok{(}\DataTypeTok{red =} \FloatTok{0.5}\NormalTok{, }\DataTypeTok{green =} \FloatTok{0.3}\NormalTok{, }\DataTypeTok{blue =} \FloatTok{0.7}\NormalTok{, }\DataTypeTok{maxColorValue =} \FloatTok{1.0}\NormalTok{)}
\end{Highlighting}
\end{Shaded}

\hypertarget{create-your-own-palette-of-rgb-colors}{%
\subsection{2.2 Create Your Own Palette of RGB
colors}\label{create-your-own-palette-of-rgb-colors}}

\begin{Shaded}
\begin{Highlighting}[]
\NormalTok{get_palette <-}\StringTok{ }\ControlFlowTok{function}\NormalTok{(}\DataTypeTok{K =} \DecValTok{0}\NormalTok{, }\DataTypeTok{n_bit =} \DecValTok{0}\NormalTok{)\{}
  \CommentTok{### warnings}
  \KeywordTok{try}\NormalTok{(}\ControlFlowTok{if}\NormalTok{(K }\OperatorTok{==}\StringTok{ }\DecValTok{0} \OperatorTok{&}\StringTok{ }\NormalTok{n_bit }\OperatorTok{==}\StringTok{ }\DecValTok{0}\NormalTok{) }\KeywordTok{stop}\NormalTok{(}\StringTok{"Type in K or n_bit"}\NormalTok{))}
  \KeywordTok{try}\NormalTok{(}\ControlFlowTok{if}\NormalTok{(K }\OperatorTok{&}\StringTok{ }\NormalTok{n_bit) }\KeywordTok{stop}\NormalTok{(}\StringTok{"Type in only K or n_bit"}\NormalTok{))}
  \KeywordTok{try}\NormalTok{(}\ControlFlowTok{if}\NormalTok{(}\KeywordTok{as.integer}\NormalTok{(K)}\OperatorTok{!=}\NormalTok{K }\OperatorTok{|}\StringTok{ }\KeywordTok{as.integer}\NormalTok{(n_bit)}\OperatorTok{!=}\StringTok{ }\NormalTok{n_bit) }\KeywordTok{stop}\NormalTok{(}\StringTok{"Wrong K or n_bit"}\NormalTok{))}
  \ControlFlowTok{if}\NormalTok{(K)\{n_bit <-}\StringTok{ }\KeywordTok{log2}\NormalTok{(K)\} }\CommentTok{# using n_bit for this function}
  \KeywordTok{try}\NormalTok{(}\ControlFlowTok{if}\NormalTok{(n_bit }\OperatorTok\StringTok{ }\DecValTok{3} \OperatorTok{!=}\StringTok{ }\DecValTok{0}\NormalTok{) }\KeywordTok{stop}\NormalTok{(}\StringTok{"Wrong K or n_bit"}\NormalTok{))}
  
  \CommentTok{### start the real work}
\NormalTok{  n <-}\StringTok{ }\DecValTok{2}\OperatorTok{^}\NormalTok{(n_bit}\OperatorTok{/}\DecValTok{3}\NormalTok{)}
\NormalTok{  col_list <-}\StringTok{ }\KeywordTok{seq}\NormalTok{(}\DecValTok{0}\NormalTok{,}\DecValTok{255}\NormalTok{,}\DataTypeTok{length.out=}\NormalTok{n)}
\NormalTok{  dat <-}\StringTok{ }\KeywordTok{expand.grid}\NormalTok{(}\DataTypeTok{Red =}\NormalTok{ col_list, }\DataTypeTok{Green =}\NormalTok{ col_list, }\DataTypeTok{Blue =}\NormalTok{ col_list)}
\NormalTok{  cols <-}\StringTok{ }\NormalTok{dat }\OperatorTok
\StringTok{    }\KeywordTok{mutate}\NormalTok{(}\DataTypeTok{cols =} \KeywordTok{rgb}\NormalTok{(Red,Green,Blue,}\DataTypeTok{maxColorValue=}\DecValTok{255}\NormalTok{)) }\OperatorTok
\StringTok{    }\KeywordTok{select}\NormalTok{(cols)}
\NormalTok{  ans <-}\StringTok{ }\KeywordTok{list}\NormalTok{(}\DataTypeTok{cols =} \KeywordTok{as.vector}\NormalTok{(}\KeywordTok{t}\NormalTok{(cols)), }\DataTypeTok{dat =}\NormalTok{ dat)}
  \KeywordTok{return}\NormalTok{(ans)}
\NormalTok{\}}
\end{Highlighting}
\end{Shaded}

\begin{Shaded}
\begin{Highlighting}[]
\NormalTok{My_RGB_03bit <-}\StringTok{ }\KeywordTok{get_palette}\NormalTok{(}\DataTypeTok{n_bit=}\DecValTok{3}\NormalTok{)}
\NormalTok{My_RGB_03bit}\OperatorTok{$}\NormalTok{dat}
\end{Highlighting}
\end{Shaded}

\hypertarget{a-naive-approach-to-color-reduction}{%
\subsection{2.3 A Naive Approach to Color
Reduction}\label{a-naive-approach-to-color-reduction}}

\hypertarget{a-compress-thepicture-into-the-colors-from-the-3-bit-rgb-palette}{%
\subsubsection{2.3.a Compress thepicture into the colors from the 3-bit
RGB
palette}\label{a-compress-thepicture-into-the-colors-from-the-3-bit-rgb-palette}}

\begin{Shaded}
\begin{Highlighting}[]
\NormalTok{close_rgb <-}\StringTok{ }\ControlFlowTok{function}\NormalTok{(red, green, blue, }\DataTypeTok{RGB=}\NormalTok{RGB_03bit) \{}
  \CommentTok{### calculate distance}
\NormalTok{  diff <-}\StringTok{ }\NormalTok{(}\KeywordTok{c}\NormalTok{(red,green,blue)}\OperatorTok{*}\DecValTok{255} \OperatorTok{+}\StringTok{ }\FloatTok{0.5}\NormalTok{)  }\OperatorTok\StringTok{ }
\StringTok{    }\KeywordTok{floor}\NormalTok{(.) }\OperatorTok\StringTok{ }
\StringTok{    }\KeywordTok{sweep}\NormalTok{(RGB}\OperatorTok{$}\NormalTok{dat, }\DecValTok{2}\NormalTok{, .)  }\CommentTok{# minus by row}
 
  \CommentTok{### find the row in RGB$dat that gives min distance}
\NormalTok{  min_row <-}\StringTok{ }\NormalTok{diff}\OperatorTok{^}\DecValTok{2}  \OperatorTok
\StringTok{    }\KeywordTok{apply}\NormalTok{(., }\DataTypeTok{MARGIN=}\DecValTok{1}\NormalTok{, sum) }\OperatorTok
\StringTok{    }\KeywordTok{which.min}\NormalTok{(.)}
 
  \CommentTok{### get rgb_color}
\NormalTok{  new_rgb <-}\StringTok{ }\NormalTok{RGB}\OperatorTok{$}\NormalTok{dat[min_row,]}\OperatorTok\StringTok{ }
\StringTok{    }\KeywordTok{mutate}\NormalTok{(}\DataTypeTok{rgb_color =} \KeywordTok{rgb}\NormalTok{(Red, Green, Blue,}\DataTypeTok{max=}\DecValTok{255}\NormalTok{))}
  \KeywordTok{return}\NormalTok{(new_rgb}\OperatorTok{$}\NormalTok{rgb_color)}
\NormalTok{\}}
\end{Highlighting}
\end{Shaded}

\begin{Shaded}
\begin{Highlighting}[]
\NormalTok{new_raster<-}\StringTok{ }\NormalTok{df_xmas }\OperatorTok
\StringTok{  }\KeywordTok{rowwise}\NormalTok{() }\OperatorTok\StringTok{ }\CommentTok{# necessary for self-defined function}
\StringTok{  }\KeywordTok{mutate}\NormalTok{(}\DataTypeTok{new_rgb_color =} \KeywordTok{close_rgb}\NormalTok{(red, green, blue)) }\OperatorTok
\StringTok{  }\KeywordTok{select}\NormalTok{(row,col,new_rgb_color) }\OperatorTok\StringTok{ }
\StringTok{  }\KeywordTok{pivot_wider}\NormalTok{(}\DataTypeTok{names_from=}\StringTok{'col'}\NormalTok{,}\DataTypeTok{values_from =} \StringTok{'new_rgb_color'}\NormalTok{)  }\OperatorTok\StringTok{ }
\StringTok{  }\KeywordTok{subset}\NormalTok{(., }\DataTypeTok{select =} \OperatorTok{-}\KeywordTok{c}\NormalTok{(row) )  }\OperatorTok\StringTok{ }
\StringTok{  }\KeywordTok{as.matrix}\NormalTok{() }\OperatorTok\StringTok{ }
\StringTok{  }\KeywordTok{as.raster}\NormalTok{()}
\end{Highlighting}
\end{Shaded}

Draw:

\begin{Shaded}
\begin{Highlighting}[]
\KeywordTok{plot}\NormalTok{(new_raster)}
\end{Highlighting}
\end{Shaded}

\clearpage

\hypertarget{floyd-steinberg-dithering-algorithm}{%
\section{3 Floyd-Steinberg dithering
algorithm}\label{floyd-steinberg-dithering-algorithm}}

\hypertarget{programming-your-own-floyd-steinberg-algorithm}{%
\subsection{3.1 Programming your own Floyd-Steinberg
algorithm}\label{programming-your-own-floyd-steinberg-algorithm}}

Create the update\_rgb function

\begin{Shaded}
\begin{Highlighting}[]
\NormalTok{update_rgb <-}\StringTok{ }\ControlFlowTok{function}\NormalTok{(x, RGB)\{}
\NormalTok{  min_row <-}\StringTok{ }\KeywordTok{sweep}\NormalTok{(RGB}\OperatorTok{$}\NormalTok{dat, }\DecValTok{2}\NormalTok{, x)}\OperatorTok{^}\DecValTok{2} \OperatorTok\StringTok{ }\CommentTok{# minus by row}
\StringTok{    }\KeywordTok{apply}\NormalTok{(., }\DecValTok{1}\NormalTok{, sum) }\OperatorTok
\StringTok{    }\KeywordTok{which.min}\NormalTok{(.)}
  \KeywordTok{return}\NormalTok{(RGB}\OperatorTok{$}\NormalTok{dat[min_row,])}
\NormalTok{\}}
\end{Highlighting}
\end{Shaded}

Create the Floyd-Steinberg function:

Two input arguments: - an array the represents the pictur, like xmas - a
matrix, like RGB\_03bit\$dat

The output : - an array of the colors with which the pixels should get
replaced - an array that holds your estimates of the diffused errors for
each pixel - the value of your loss function

\begin{Shaded}
\begin{Highlighting}[]
\NormalTok{Floyd_Steinberg <-}\StringTok{ }\ControlFlowTok{function}\NormalTok{(pic, RGB)\{}
\NormalTok{  err <-}\StringTok{ }\NormalTok{pic }\CommentTok{# error matrix, update later}
\NormalTok{  nrow <-}\StringTok{ }\KeywordTok{dim}\NormalTok{(pic)[}\DecValTok{1}\NormalTok{]}
\NormalTok{  ncol <-}\StringTok{ }\KeywordTok{dim}\NormalTok{(pic)[}\DecValTok{2}\NormalTok{]}
\NormalTok{  pic <-}\StringTok{ }\KeywordTok{floor}\NormalTok{(pic}\OperatorTok{*}\DecValTok{255}\FloatTok{+0.5}\NormalTok{)}
  \ControlFlowTok{for}\NormalTok{ (r }\ControlFlowTok{in} \DecValTok{1}\OperatorTok{:}\NormalTok{nrow)\{}
     \ControlFlowTok{for}\NormalTok{ (c }\ControlFlowTok{in} \DecValTok{1}\OperatorTok{:}\NormalTok{ncol)\{}
\NormalTok{       x <-}\StringTok{ }\NormalTok{pic[r,c,]   }\CommentTok{#old pixel}
\NormalTok{       pic[r,c,] <-}\StringTok{ }\KeywordTok{update_rgb}\NormalTok{(x,RGB) }\OperatorTok\StringTok{ }\KeywordTok{unlist}\NormalTok{(.) }\OperatorTok\StringTok{ }\KeywordTok{as.vector}\NormalTok{(.) }\CommentTok{#new pixel}
\NormalTok{       err[r,c,] <-}\StringTok{ }\NormalTok{x }\OperatorTok{-}\StringTok{ }\NormalTok{pic[r,c,] }\CommentTok{#quant_error}
       \KeywordTok{try}\NormalTok{(pic[r  ,c}\OperatorTok{+}\DecValTok{1}\NormalTok{,] <-}\StringTok{ }\NormalTok{pic[r  ,c}\OperatorTok{+}\DecValTok{1}\NormalTok{,] }\OperatorTok{+}\StringTok{ }\NormalTok{err[r,c,] }\OperatorTok{*}\StringTok{ }\DecValTok{7} \OperatorTok{/}\StringTok{ }\DecValTok{16}\NormalTok{) }\CommentTok{#ignore error}
       \KeywordTok{try}\NormalTok{(pic[r}\OperatorTok{+}\DecValTok{1}\NormalTok{,c}\DecValTok{-1}\NormalTok{,] <-}\StringTok{ }\NormalTok{pic[r}\OperatorTok{+}\DecValTok{1}\NormalTok{,c}\DecValTok{-1}\NormalTok{,] }\OperatorTok{+}\StringTok{ }\NormalTok{err[r,c,] }\OperatorTok{*}\StringTok{ }\DecValTok{3} \OperatorTok{/}\StringTok{ }\DecValTok{16}\NormalTok{)}
       \KeywordTok{try}\NormalTok{(pic[r}\OperatorTok{+}\DecValTok{1}\NormalTok{,c  ,] <-}\StringTok{ }\NormalTok{pic[r}\OperatorTok{+}\DecValTok{1}\NormalTok{,c  ,] }\OperatorTok{+}\StringTok{ }\NormalTok{err[r,c,] }\OperatorTok{*}\StringTok{ }\DecValTok{5} \OperatorTok{/}\StringTok{ }\DecValTok{16}\NormalTok{)}
       \KeywordTok{try}\NormalTok{(pic[r}\OperatorTok{+}\DecValTok{1}\NormalTok{,c}\OperatorTok{+}\DecValTok{1}\NormalTok{,] <-}\StringTok{ }\NormalTok{pic[r}\OperatorTok{+}\DecValTok{1}\NormalTok{,c}\OperatorTok{+}\DecValTok{1}\NormalTok{,] }\OperatorTok{+}\StringTok{ }\NormalTok{err[r,c,] }\OperatorTok{*}\StringTok{ }\DecValTok{1} \OperatorTok{/}\StringTok{ }\DecValTok{16}\NormalTok{)}
\NormalTok{     \} }
\NormalTok{  \}}
\NormalTok{  pic =}\StringTok{ }\NormalTok{pic}\OperatorTok{/}\DecValTok{255}
\NormalTok{  loss <-}\StringTok{ }\KeywordTok{sum}\NormalTok{(err}\OperatorTok{^}\DecValTok{2}\NormalTok{)}
  \KeywordTok{return}\NormalTok{(}\KeywordTok{list}\NormalTok{(}\DataTypeTok{img =}\NormalTok{ pic, }\DataTypeTok{err_mat =}\NormalTok{ err, }\DataTypeTok{loss =}\NormalTok{ loss))}
\NormalTok{\}}
\end{Highlighting}
\end{Shaded}

\begin{Shaded}
\begin{Highlighting}[]
\NormalTok{dither_ans_}\DecValTok{03}\NormalTok{ <-}\StringTok{ }\KeywordTok{Floyd_Steinberg}\NormalTok{(xmas, RGB_03bit) }
\end{Highlighting}
\end{Shaded}

\begin{Shaded}
\begin{Highlighting}[]
\KeywordTok{plot}\NormalTok{(}\OtherTok{NULL}\NormalTok{, }\DataTypeTok{xlim=}\KeywordTok{c}\NormalTok{(}\DecValTok{0}\NormalTok{, }\DecValTok{1}\NormalTok{), }\DataTypeTok{ylim=}\KeywordTok{c}\NormalTok{(}\DecValTok{0}\NormalTok{, }\DecValTok{1}\NormalTok{),}\DataTypeTok{axes=}\OtherTok{FALSE}\NormalTok{, }\DataTypeTok{frame.plot=}\OtherTok{TRUE}\NormalTok{, }\DataTypeTok{ann =} \OtherTok{FALSE}\NormalTok{)}
\NormalTok{grid}\OperatorTok{::}\KeywordTok{grid.raster}\NormalTok{(dither_ans_}\DecValTok{03}\OperatorTok{$}\NormalTok{img)}
\end{Highlighting}
\end{Shaded}

\hypertarget{plotting-the-loss-for-3-bit-9-bit-and-15-bit}{%
\subsection{3.2 Plotting the Loss for 3-bit, 9-bit, and
15-bit}\label{plotting-the-loss-for-3-bit-9-bit-and-15-bit}}

\begin{Shaded}
\begin{Highlighting}[]
\NormalTok{los <-}\StringTok{ }\KeywordTok{c}\NormalTok{(dither_ans_}\DecValTok{03}\OperatorTok{$}\NormalTok{loss,}\KeywordTok{sum}\NormalTok{(dither_09bit}\OperatorTok{$}\NormalTok{err_mat}\OperatorTok{^}\DecValTok{2}\NormalTok{),}\KeywordTok{sum}\NormalTok{(dither_15bit}\OperatorTok{$}\NormalTok{err_mat}\OperatorTok{^}\DecValTok{2}\NormalTok{))}
\KeywordTok{plot}\NormalTok{(}\KeywordTok{c}\NormalTok{(}\DecValTok{3}\NormalTok{,}\DecValTok{9}\NormalTok{,}\DecValTok{15}\NormalTok{),}\KeywordTok{log}\NormalTok{(los),}\DataTypeTok{type=}\StringTok{'l'}\NormalTok{,}\DataTypeTok{xlab=}\StringTok{'n_bit'}\NormalTok{)}
\KeywordTok{points}\NormalTok{(}\KeywordTok{c}\NormalTok{(}\DecValTok{3}\NormalTok{,}\DecValTok{9}\NormalTok{,}\DecValTok{15}\NormalTok{),}\KeywordTok{log}\NormalTok{(los))}
\end{Highlighting}
\end{Shaded}

\clearpage

\hypertarget{statistical-computing-on-the-floyd-steinberg-algorithm}{%
\section{4 Statistical Computing on the Floyd-Steinberg
algorithm}\label{statistical-computing-on-the-floyd-steinberg-algorithm}}

\hypertarget{generate-a-permutation-of-the-picture}{%
\subsection{4.1 Generate a Permutation of the
Picture}\label{generate-a-permutation-of-the-picture}}

Create a function that produces a permuted replicate of the
\texttt{xmas} variable, denoted by \(\mathcal{X}^b\). Each pixel
\(\mathbf{x}^b_{ij}\) in \(\mathcal{X}^b\) is a realization of a
permutation over \(i\) and \(j\) of the pixels in \(\mathcal{X}\).

\begin{Shaded}
\begin{Highlighting}[]
\NormalTok{create_a_perm <-}\StringTok{ }\ControlFlowTok{function}\NormalTok{(xmas,B)\{}
  \KeywordTok{set.seed}\NormalTok{(B)}
  
\NormalTok{  nrow <-}\StringTok{ }\KeywordTok{dim}\NormalTok{(xmas)[}\DecValTok{1}\NormalTok{]}
\NormalTok{  ncol <-}\StringTok{ }\KeywordTok{dim}\NormalTok{(xmas)[}\DecValTok{2}\NormalTok{]}
\NormalTok{  n <-}\StringTok{ }\NormalTok{nrow }\OperatorTok{*}\StringTok{  }\NormalTok{ncol }
\NormalTok{  ind <-}\StringTok{ }\KeywordTok{sample}\NormalTok{(}\DecValTok{1}\OperatorTok{:}\NormalTok{n, n, }\DataTypeTok{replace =} \OtherTok{FALSE}\NormalTok{)}
\NormalTok{  perm_pic <-}\StringTok{ }\NormalTok{xmas }\CommentTok{# make a copy, update it later}
  
  \ControlFlowTok{for}\NormalTok{ (i }\ControlFlowTok{in} \DecValTok{1}\OperatorTok{:}\NormalTok{n)\{}
\NormalTok{    row_ind <-}\StringTok{ }\KeywordTok{ceiling}\NormalTok{(ind[i]}\OperatorTok{/}\NormalTok{ncol)}
\NormalTok{    col_ind <-}\StringTok{ }\NormalTok{ind[i] }\OperatorTok\StringTok{ }\NormalTok{ncol }\OperatorTok{+}\StringTok{ }\DecValTok{1}
\NormalTok{    perm_pic[row_ind,col_ind,] <-}\StringTok{ }\NormalTok{xmas[}\KeywordTok{ceiling}\NormalTok{(i}\OperatorTok{/}\NormalTok{ncol),i }\OperatorTok\StringTok{ }\NormalTok{ncol}\OperatorTok{+}\DecValTok{1}\NormalTok{,]}
\NormalTok{  \}}
  \KeywordTok{return}\NormalTok{(perm_pic)}
\NormalTok{\}}
\CommentTok{#perm_pic<-create_a_perm(xmas,2020)}
\CommentTok{#plot(NULL, xlim=c(0, 1), ylim=c(0, 1),axes=FALSE, frame.plot=TRUE, ann = FALSE)}
\CommentTok{#grid::grid.raster(perm_pic)}
\end{Highlighting}
\end{Shaded}

\hypertarget{log-loss-of-floyd-steinberg-under-h0}{%
\subsection{4.2 Log Loss of Floyd-Steinberg under
H0}\label{log-loss-of-floyd-steinberg-under-h0}}

Write your own function that outputs a variable like
xmas\_replicates\_logloss. input: - an array of an image (like xmas) - a
vector of values for K\_values that can only belong to the set 2\^{}(3 *
(1:5)) - B the number of replicates that need to be created.

\begin{Shaded}
\begin{Highlighting}[]
\NormalTok{create_perm_logloss <-}\StringTok{ }\ControlFlowTok{function}\NormalTok{(xmas,bits,B)\{}
\NormalTok{  RGB <-}\StringTok{ }\NormalTok{parallel}\OperatorTok{::}\KeywordTok{mclapply}\NormalTok{(bits,get_palette) }\CommentTok{# Create palettes}
\NormalTok{  K_n <-}\StringTok{ }\KeywordTok{length}\NormalTok{(bits)}
\NormalTok{  logloss_list <-}\StringTok{ }\NormalTok{parallel}\OperatorTok{::}\KeywordTok{mclapply}\NormalTok{(}\DecValTok{1}\OperatorTok{:}\NormalTok{B, }\ControlFlowTok{function}\NormalTok{(x) \{}
\NormalTok{    perm_pic <-}\StringTok{ }\KeywordTok{create_a_perm}\NormalTok{(xmas,x)}
\NormalTok{    logloss <-}\StringTok{ }\NormalTok{parallel}\OperatorTok{::}\KeywordTok{mclapply}\NormalTok{(}\DecValTok{1}\OperatorTok{:}\NormalTok{K_n, }\ControlFlowTok{function}\NormalTok{(y)\{}
\NormalTok{      ans <-}\StringTok{ }\KeywordTok{Floyd_Steinberg}\NormalTok{(perm_pic, RGB[[y]])}
      \KeywordTok{return}\NormalTok{(}\KeywordTok{list}\NormalTok{(}\DataTypeTok{bit =} \DecValTok{3}\OperatorTok{*}\NormalTok{bits[y], }\DataTypeTok{logloss=}\NormalTok{ans}\OperatorTok{$}\NormalTok{loss))}
\NormalTok{      \})}
    \KeywordTok{return}\NormalTok{(logloss)}
\NormalTok{    \})}
  \KeywordTok{return}\NormalTok{(logloss_list)}
\NormalTok{  \}}
\end{Highlighting}
\end{Shaded}

\begin{Shaded}
\begin{Highlighting}[]
\CommentTok{### Test if the function works}
\NormalTok{B <-}\StringTok{ }\DecValTok{2}
\NormalTok{bits <-}\StringTok{ }\DecValTok{2}\OperatorTok{^}\NormalTok{(}\DecValTok{3} \OperatorTok{*}\StringTok{ }\NormalTok{(}\DecValTok{1}\OperatorTok{:}\DecValTok{5}\NormalTok{))}
\NormalTok{ans <-}\StringTok{ }\KeywordTok{create_perm_logloss}\NormalTok{(xmas[}\DecValTok{1}\OperatorTok{:}\DecValTok{10}\NormalTok{,}\DecValTok{1}\OperatorTok{:}\DecValTok{10}\NormalTok{,],bits,B)}
\end{Highlighting}
\end{Shaded}

\hypertarget{visualize-the-log-loss-under-h_0-and-for-our-data}{%
\subsection{4.4 Visualize the Log Loss under H\_0 and for our
data}\label{visualize-the-log-loss-under-h_0-and-for-our-data}}

To get an estimate for the expected value of the loss function for each
RGB palette under H0:

\begin{Shaded}
\begin{Highlighting}[]
\NormalTok{B <-}\StringTok{ }\KeywordTok{length}\NormalTok{(xmas_replicates_logloss)}
\NormalTok{K_n <-}\StringTok{ }\KeywordTok{length}\NormalTok{(xmas_replicates_logloss[[}\DecValTok{1}\NormalTok{]])}
\NormalTok{log_loss_k <-}\StringTok{ }\KeywordTok{matrix}\NormalTok{(}\DecValTok{0}\NormalTok{,}\DataTypeTok{nrow=}\NormalTok{B,}\DataTypeTok{ncol=}\NormalTok{K_n)}
\ControlFlowTok{for}\NormalTok{(i }\ControlFlowTok{in} \DecValTok{1}\OperatorTok{:}\NormalTok{B)\{}
  \ControlFlowTok{for}\NormalTok{(j }\ControlFlowTok{in} \DecValTok{1}\OperatorTok{:}\NormalTok{K_n)\{}
\NormalTok{    log_loss_k[i,j]<-}\StringTok{ }\NormalTok{xmas_replicates_logloss[[i]][[j]]}\OperatorTok{$}\NormalTok{logloss}
\NormalTok{  \}}
\NormalTok{\} }

\NormalTok{log_loss_perm <-}\StringTok{ }\KeywordTok{colMeans}\NormalTok{(log_loss_k)}
\end{Highlighting}
\end{Shaded}

Compute the difference:

\begin{Shaded}
\begin{Highlighting}[]
\NormalTok{dither_list <-}\StringTok{ }\KeywordTok{list}\NormalTok{(dither_03bit,dither_06bit,dither_09bit,dither_12bit,dither_15bit)}
\NormalTok{loglos_true <-}\StringTok{ }\KeywordTok{sapply}\NormalTok{(dither_list, }\ControlFlowTok{function}\NormalTok{(x) }\KeywordTok{log}\NormalTok{(}\KeywordTok{sum}\NormalTok{(x}\OperatorTok{$}\NormalTok{err_mat}\OperatorTok{^}\DecValTok{2}\NormalTok{)))}

\NormalTok{diff <-}\StringTok{ }\NormalTok{log_loss_perm }\OperatorTok{-}\StringTok{ }\NormalTok{loglos_true}
\end{Highlighting}
\end{Shaded}

Best bit:

\begin{Shaded}
\begin{Highlighting}[]
\KeywordTok{which.max}\NormalTok{(diff) }\OperatorTok\StringTok{ }\KeywordTok{c}\NormalTok{(}\DecValTok{3}\NormalTok{,}\DecValTok{6}\NormalTok{,}\DecValTok{9}\NormalTok{,}\DecValTok{12}\NormalTok{,}\DecValTok{15}\NormalTok{)[.] }
\end{Highlighting}
\end{Shaded}

Create the data format for ggplot

\begin{Shaded}
\begin{Highlighting}[]
\NormalTok{df <-}\StringTok{ }\KeywordTok{data.frame}\NormalTok{(}\DataTypeTok{bit =} \KeywordTok{c}\NormalTok{(}\DecValTok{3}\NormalTok{,}\DecValTok{6}\NormalTok{,}\DecValTok{9}\NormalTok{,}\DecValTok{12}\NormalTok{,}\DecValTok{15}\NormalTok{), }\DataTypeTok{h_0 =}\NormalTok{log_loss_perm, }\DataTypeTok{observed =}\NormalTok{ loglos_true)}
\NormalTok{df_longer <-}\StringTok{ }\KeywordTok{pivot_longer}\NormalTok{(df,}\DataTypeTok{cols=}\KeywordTok{c}\NormalTok{(}\StringTok{'h_0'}\NormalTok{,}\StringTok{'observed'}\NormalTok{), }\DataTypeTok{names_to =} \StringTok{'type'}\NormalTok{, }\DataTypeTok{values_to =}\StringTok{'log_loss'}\NormalTok{)}

\NormalTok{std <-}\StringTok{ }\KeywordTok{sqrt}\NormalTok{((}\DecValTok{1}\OperatorTok{+}\DecValTok{1}\OperatorTok{/}\NormalTok{B)) }\OperatorTok{*}\StringTok{ }\KeywordTok{apply}\NormalTok{(log_loss_k, }\DecValTok{2}\NormalTok{, sd)}
\NormalTok{df2<-}\KeywordTok{data.frame}\NormalTok{(}\DataTypeTok{bit =} \KeywordTok{c}\NormalTok{(}\DecValTok{3}\NormalTok{,}\DecValTok{6}\NormalTok{,}\DecValTok{9}\NormalTok{,}\DecValTok{12}\NormalTok{,}\DecValTok{15}\NormalTok{), }\DataTypeTok{gap =}\NormalTok{ diff, }\DataTypeTok{std =}\NormalTok{ std)}
\end{Highlighting}
\end{Shaded}

Draw:

\begin{Shaded}
\begin{Highlighting}[]
\NormalTok{p1<-}\KeywordTok{ggplot}\NormalTok{(}\KeywordTok{aes}\NormalTok{(}\DataTypeTok{x=}\NormalTok{bit  , }\DataTypeTok{y=}\NormalTok{ log_loss) , }\DataTypeTok{data =}\NormalTok{ df_longer) }\OperatorTok{+}
\StringTok{  }\KeywordTok{geom_line}\NormalTok{(}\KeywordTok{aes}\NormalTok{(}\DataTypeTok{color =}\NormalTok{ type)) }\OperatorTok{+}
\StringTok{  }\KeywordTok{geom_point}\NormalTok{(}\KeywordTok{aes}\NormalTok{(}\DataTypeTok{fill=}\NormalTok{type), }\DataTypeTok{size=}\DecValTok{4}\NormalTok{, }\DataTypeTok{shape=}\DecValTok{21}\NormalTok{, }\DataTypeTok{color=}\StringTok{'transparent'}\NormalTok{) }\OperatorTok{+}
\StringTok{  }\KeywordTok{ggtitle}\NormalTok{(}\StringTok{'Loss: H0 and Observed Data'}\NormalTok{) }\OperatorTok{+}
\StringTok{  }\KeywordTok{scale_fill_manual}\NormalTok{(}\DataTypeTok{values=}\KeywordTok{c}\NormalTok{(}\StringTok{'red'}\NormalTok{,}\StringTok{'blue'}\NormalTok{)) }\OperatorTok{+}\StringTok{ }
\StringTok{  }\KeywordTok{scale_color_manual}\NormalTok{(}\DataTypeTok{values=}\KeywordTok{c}\NormalTok{(}\StringTok{'red'}\NormalTok{,}\StringTok{'blue'}\NormalTok{)) }\OperatorTok{+}\StringTok{ }
\StringTok{  }\KeywordTok{theme_bw}\NormalTok{()}\OperatorTok{+}
\StringTok{  }\KeywordTok{theme}\NormalTok{(}\DataTypeTok{plot.title =} \KeywordTok{element_text}\NormalTok{(}\DataTypeTok{hjust =} \FloatTok{0.5}\NormalTok{,}\DataTypeTok{face=}\StringTok{'bold'}\NormalTok{))}

\NormalTok{p2<-}\KeywordTok{ggplot}\NormalTok{(}\KeywordTok{aes}\NormalTok{(}\DataTypeTok{x=}\NormalTok{bit  , }\DataTypeTok{y=}\NormalTok{ gap) , }\DataTypeTok{data =}\NormalTok{ df2) }\OperatorTok{+}
\StringTok{  }\KeywordTok{geom_line}\NormalTok{() }\OperatorTok{+}
\StringTok{  }\KeywordTok{geom_errorbar}\NormalTok{(}\KeywordTok{aes}\NormalTok{(}\DataTypeTok{ymin=}\NormalTok{gap}\DecValTok{-2}\OperatorTok{*}\NormalTok{std, }\DataTypeTok{ymax=}\NormalTok{gap}\OperatorTok{+}\DecValTok{2}\OperatorTok{*}\NormalTok{std), }\DataTypeTok{width=}\FloatTok{2.5}\NormalTok{, }\DataTypeTok{color =} \StringTok{'red'}\NormalTok{, }\DataTypeTok{linetype=}\DecValTok{2}\NormalTok{) }\OperatorTok{+}\StringTok{ }
\StringTok{  }\KeywordTok{geom_point}\NormalTok{(}\DataTypeTok{fill =} \StringTok{'black'}\NormalTok{, }\DataTypeTok{size=}\DecValTok{4}\NormalTok{, }\DataTypeTok{shape=}\DecValTok{21}\NormalTok{) }\OperatorTok{+}
\StringTok{  }\KeywordTok{ggtitle}\NormalTok{(}\StringTok{'Results Gap Statistic'}\NormalTok{) }\OperatorTok{+}
\StringTok{  }\KeywordTok{theme_bw}\NormalTok{() }\OperatorTok{+}
\StringTok{  }\KeywordTok{theme}\NormalTok{(}\DataTypeTok{plot.title =} \KeywordTok{element_text}\NormalTok{(}\DataTypeTok{hjust =} \FloatTok{0.5}\NormalTok{,}\DataTypeTok{face=}\StringTok{'bold'}\NormalTok{))}\OperatorTok{+}
\StringTok{  }\KeywordTok{scale_x_continuous}\NormalTok{(}\DataTypeTok{limits=}\KeywordTok{c}\NormalTok{(}\DecValTok{0}\NormalTok{, }\DecValTok{18}\NormalTok{),}\DataTypeTok{breaks=}\KeywordTok{seq}\NormalTok{(}\DecValTok{3}\NormalTok{,}\DecValTok{15}\NormalTok{,}\DecValTok{3}\NormalTok{))}

\NormalTok{gridExtra}\OperatorTok{::}\KeywordTok{grid.arrange}\NormalTok{(p1,p2,}\DataTypeTok{ncol=}\DecValTok{2}\NormalTok{)}
\end{Highlighting}
\end{Shaded}

\hypertarget{alternatives}{%
\subsection{4.6 Alternatives}\label{alternatives}}

Could you explain why the GAP statistic is small for too small K, and
also small for too large K? For your explanation, relate to the bias
variance trade-off.

Ans: - When K is small, there are not many color options. The RGB platte
can only provide rough approximation of the original colors. This leads
to high \emph{bias}. Therefore, both permutation and original picture
will have large but close log loss. Close log loss will give a small GAP
statistic. - However, too many color options will cuase high
\emph{variancce} since a tiny change in the original color may lead to a
different RGB platte color. Therefore, both permutation and original
picture will have small but close log loss. Close log loss will give a
small GAP statistic.

\clearpage

\hypertarget{bonus-something-new-the-package-rcpp-15-points}{%
\section{\texorpdfstring{5. Bonus: Something new, the package
\texttt{Rcpp} (15
points)}{5. Bonus: Something new, the package Rcpp (15 points)}}\label{bonus-something-new-the-package-rcpp-15-points}}

\begin{Shaded}
\begin{Highlighting}[]
\KeywordTok{cppFunction}\NormalTok{(}\StringTok{'int***  Floyd_Steinberg_cpp(int*** pic,int height,int width) \{}
\StringTok{  int h = width, int w = width;}
\StringTok{  for (int x = 1; x < h-1; x ++) \{}
\StringTok{      for (int y = 2; j < w; y++) \{}
\StringTok{          int oldPixel = pic[x][y];}
\StringTok{          int newPixel = find_closest_palette_color(oldPixel);}
\StringTok{          pic[x][y] = newPixel;}
\StringTok{          int quant_error = oldPixel - newPixel;}
\StringTok{          pic[x 1][y+1] = pic[x 1][y+1] + (quant_error * 7/16);}
\StringTok{          pic[x+1][y-1] = pic[x+1][y-1] + (quant_error * 3/16);}
\StringTok{          pic[x+1][y  ] = pic[x+1][y  ] + (quant_error * 5/16);}
\StringTok{          pic[x+1][y+1] = pic[x+1][y+1] + (quant_error * 1/16);}
\StringTok{      \}}
\StringTok{  \}}
\StringTok{  return pic;}
\StringTok{\}'}\NormalTok{)}


\NormalTok{ new_pic <-}\StringTok{ }\KeywordTok{Floyd_Steinberg_cpp}\NormalTok{(xmas}\OperatorTok{*}\DecValTok{255}\NormalTok{,}\KeywordTok{dim}\NormalTok{(xmas)[}\DecValTok{1}\NormalTok{],}\KeywordTok{dim}\NormalTok{(xmas)[}\DecValTok{2}\NormalTok{])}
\end{Highlighting}
\end{Shaded}


\end{document}

\documentclass[]{article}
\usepackage{lmodern}
\usepackage{amssymb,amsmath}
\usepackage{ifxetex,ifluatex}
\usepackage{fixltx2e} % provides \textsubscript
\ifnum 0\ifxetex 1\fi\ifluatex 1\fi=0 % if pdftex
  \usepackage[T1]{fontenc}
  \usepackage[utf8]{inputenc}
\else % if luatex or xelatex
  \ifxetex
    \usepackage{mathspec}
  \else
    \usepackage{fontspec}
  \fi
  \defaultfontfeatures{Ligatures=TeX,Scale=MatchLowercase}
\fi
% use upquote if available, for straight quotes in verbatim environments
\IfFileExists{upquote.sty}{\usepackage{upquote}}{}
% use microtype if available
\IfFileExists{microtype.sty}{%
\usepackage{microtype}
\UseMicrotypeSet[protrusion]{basicmath} % disable protrusion for tt fonts
}{}
\usepackage[margin=1in]{geometry}
\usepackage{hyperref}
\hypersetup{unicode=true,
            pdftitle={SCR Week 08 Exercises},
            pdfauthor={R-team},
            pdfborder={0 0 0},
            breaklinks=true}
\urlstyle{same}  % don't use monospace font for urls
\usepackage{color}
\usepackage{fancyvrb}
\newcommand{\VerbBar}{|}
\newcommand{\VERB}{\Verb[commandchars=\\\{\}]}
\DefineVerbatimEnvironment{Highlighting}{Verbatim}{commandchars=\\\{\}}
% Add ',fontsize=\small' for more characters per line
\usepackage{framed}
\definecolor{shadecolor}{RGB}{248,248,248}
\newenvironment{Shaded}{\begin{snugshade}}{\end{snugshade}}
\newcommand{\AlertTok}[1]{\textcolor[rgb]{0.94,0.16,0.16}{#1}}
\newcommand{\AnnotationTok}[1]{\textcolor[rgb]{0.56,0.35,0.01}{\textbf{\textit{#1}}}}
\newcommand{\AttributeTok}[1]{\textcolor[rgb]{0.77,0.63,0.00}{#1}}
\newcommand{\BaseNTok}[1]{\textcolor[rgb]{0.00,0.00,0.81}{#1}}
\newcommand{\BuiltInTok}[1]{#1}
\newcommand{\CharTok}[1]{\textcolor[rgb]{0.31,0.60,0.02}{#1}}
\newcommand{\CommentTok}[1]{\textcolor[rgb]{0.56,0.35,0.01}{\textit{#1}}}
\newcommand{\CommentVarTok}[1]{\textcolor[rgb]{0.56,0.35,0.01}{\textbf{\textit{#1}}}}
\newcommand{\ConstantTok}[1]{\textcolor[rgb]{0.00,0.00,0.00}{#1}}
\newcommand{\ControlFlowTok}[1]{\textcolor[rgb]{0.13,0.29,0.53}{\textbf{#1}}}
\newcommand{\DataTypeTok}[1]{\textcolor[rgb]{0.13,0.29,0.53}{#1}}
\newcommand{\DecValTok}[1]{\textcolor[rgb]{0.00,0.00,0.81}{#1}}
\newcommand{\DocumentationTok}[1]{\textcolor[rgb]{0.56,0.35,0.01}{\textbf{\textit{#1}}}}
\newcommand{\ErrorTok}[1]{\textcolor[rgb]{0.64,0.00,0.00}{\textbf{#1}}}
\newcommand{\ExtensionTok}[1]{#1}
\newcommand{\FloatTok}[1]{\textcolor[rgb]{0.00,0.00,0.81}{#1}}
\newcommand{\FunctionTok}[1]{\textcolor[rgb]{0.00,0.00,0.00}{#1}}
\newcommand{\ImportTok}[1]{#1}
\newcommand{\InformationTok}[1]{\textcolor[rgb]{0.56,0.35,0.01}{\textbf{\textit{#1}}}}
\newcommand{\KeywordTok}[1]{\textcolor[rgb]{0.13,0.29,0.53}{\textbf{#1}}}
\newcommand{\NormalTok}[1]{#1}
\newcommand{\OperatorTok}[1]{\textcolor[rgb]{0.81,0.36,0.00}{\textbf{#1}}}
\newcommand{\OtherTok}[1]{\textcolor[rgb]{0.56,0.35,0.01}{#1}}
\newcommand{\PreprocessorTok}[1]{\textcolor[rgb]{0.56,0.35,0.01}{\textit{#1}}}
\newcommand{\RegionMarkerTok}[1]{#1}
\newcommand{\SpecialCharTok}[1]{\textcolor[rgb]{0.00,0.00,0.00}{#1}}
\newcommand{\SpecialStringTok}[1]{\textcolor[rgb]{0.31,0.60,0.02}{#1}}
\newcommand{\StringTok}[1]{\textcolor[rgb]{0.31,0.60,0.02}{#1}}
\newcommand{\VariableTok}[1]{\textcolor[rgb]{0.00,0.00,0.00}{#1}}
\newcommand{\VerbatimStringTok}[1]{\textcolor[rgb]{0.31,0.60,0.02}{#1}}
\newcommand{\WarningTok}[1]{\textcolor[rgb]{0.56,0.35,0.01}{\textbf{\textit{#1}}}}
\usepackage{graphicx,grffile}
\makeatletter
\def\maxwidth{\ifdim\Gin@nat@width>\linewidth\linewidth\else\Gin@nat@width\fi}
\def\maxheight{\ifdim\Gin@nat@height>\textheight\textheight\else\Gin@nat@height\fi}
\makeatother
% Scale images if necessary, so that they will not overflow the page
% margins by default, and it is still possible to overwrite the defaults
% using explicit options in \includegraphics[width, height, ...]{}
\setkeys{Gin}{width=\maxwidth,height=\maxheight,keepaspectratio}
\IfFileExists{parskip.sty}{%
\usepackage{parskip}
}{% else
\setlength{\parindent}{0pt}
\setlength{\parskip}{6pt plus 2pt minus 1pt}
}
\setlength{\emergencystretch}{3em}  % prevent overfull lines
\providecommand{\tightlist}{%
  \setlength{\itemsep}{0pt}\setlength{\parskip}{0pt}}
\setcounter{secnumdepth}{0}
% Redefines (sub)paragraphs to behave more like sections
\ifx\paragraph\undefined\else
\let\oldparagraph\paragraph
\renewcommand{\paragraph}[1]{\oldparagraph{#1}\mbox{}}
\fi
\ifx\subparagraph\undefined\else
\let\oldsubparagraph\subparagraph
\renewcommand{\subparagraph}[1]{\oldsubparagraph{#1}\mbox{}}
\fi

%%% Use protect on footnotes to avoid problems with footnotes in titles
\let\rmarkdownfootnote\footnote%
\def\footnote{\protect\rmarkdownfootnote}

%%% Change title format to be more compact
\usepackage{titling}

% Create subtitle command for use in maketitle
\providecommand{\subtitle}[1]{
  \posttitle{
    \begin{center}\large#1\end{center}
    }
}

\setlength{\droptitle}{-2em}

  \title{SCR Week 08 Exercises}
    \pretitle{\vspace{\droptitle}\centering\huge}
  \posttitle{\par}
    \author{R-team}
    \preauthor{\centering\large\emph}
  \postauthor{\par}
      \predate{\centering\large\emph}
  \postdate{\par}
    \date{07 November, 2019}


\begin{document}
\maketitle

\hypertarget{exercises-part-1}{%
\section{Exercises part 1}\label{exercises-part-1}}

\hypertarget{tutorial-of-combinations-and-permutations}{%
\subsection{1.1 Tutorial of Combinations and
Permutations}\label{tutorial-of-combinations-and-permutations}}

In this section, we reflect upon some typical combination and
permutation problems, the corresponding mathematical notation, and the
implementation in R. In the first half, we will show two examples of
combination and permutation of a vector, without using replacement. This
means that each element of a vector may occur only once (it is not
repeated). In the second half, we will show examples with replacement.

For more information, see
\href{http://www.mathsisfun.com/combinatorics/combinations-permutations.html}{combinations
and permutations.}

\hypertarget{a-combination-1}{%
\subsubsection{a: Combination 1}\label{a-combination-1}}

How many different committees of 4 (\(r = 4\)) students can be chosen
from a group of 5 (\(n = 5\)) students? Make sure your answer is
verified with \texttt{R} code.

\emph{Hint: when writing your \LaTeX, you may like the code
\texttt{\textbackslash{}binom\{\}\{\}} or
\texttt{\{\textbackslash{}choose\}}.}

\hypertarget{b-combination-2}{%
\subsubsection{b: Combination 2}\label{b-combination-2}}

Take a look at examples of the helpfile of the function
\texttt{utils::combn}. Can you create the whole ``sample space'' for all
combinations of 4 students out of the 5 available students? Store it
inside a variable.

\begin{Shaded}
\begin{Highlighting}[]
\NormalTok{all_students <-}\StringTok{ }\KeywordTok{c}\NormalTok{(}\StringTok{"Xinru"}\NormalTok{, }\StringTok{"Ionica"}\NormalTok{, }\StringTok{"Elise"}\NormalTok{, }\StringTok{"Maryam"}\NormalTok{, }\StringTok{"Gina"}\NormalTok{) }
\end{Highlighting}
\end{Shaded}

\hypertarget{c-permutation-1}{%
\subsubsection{c: Permutation 1}\label{c-permutation-1}}

In how many ways can we permute a vector with 5 elements?

\hypertarget{c-permutation-2}{%
\subsubsection{c: Permutation 2}\label{c-permutation-2}}

Could you use the function \texttt{sample()} to draw a random
permutation out of all possible permutations of the variable
\texttt{all\_students}?

\hypertarget{d-permutation-3}{%
\subsubsection{d: Permutation 3}\label{d-permutation-3}}

Store 4 random draws (with replacement) from all possible permutations
of \texttt{all\_students} in a \texttt{list} or \texttt{matrix}.

\hypertarget{e-permutation-and-combination-with-replacement}{%
\subsubsection{e: Permutation and combination with
replacement}\label{e-permutation-and-combination-with-replacement}}

How many permutations are possible of a vector of 5 elements, when we
allow for repetition, i.e.~the same student can occur more often than
once.

NB. For your answer no \texttt{R} code is needed.

\hypertarget{f-permutation-and-combination-with-replacement}{%
\subsubsection{f: Permutation and combination with
replacement}\label{f-permutation-and-combination-with-replacement}}

In this R-code, we use a the function \texttt{expand.grid()}. This
function creates a data frame from all combinations of the categories of
the supplied vectors or factors. We will first give a small illustration
of this function:

\begin{Shaded}
\begin{Highlighting}[]
\NormalTok{sunsetcolor <-}\StringTok{ }\KeywordTok{c}\NormalTok{(}\StringTok{"red"}\NormalTok{, }\StringTok{"orange"}\NormalTok{, }\StringTok{"yellow"}\NormalTok{)}
\KeywordTok{expand.grid}\NormalTok{(sunsetcolor, sunsetcolor)}
\end{Highlighting}
\end{Shaded}

\begin{verbatim}
##     Var1   Var2
## 1    red    red
## 2 orange    red
## 3 yellow    red
## 4    red orange
## 5 orange orange
## 6 yellow orange
## 7    red yellow
## 8 orange yellow
## 9 yellow yellow
\end{verbatim}

Here, all possibilities of 2 colors are shown, and repetition of the
same color is allowed.

Use \texttt{expand.grid()} to show all possibilities there are for 5
student names using \texttt{all\_students}.

Note that the first argument(s) in \texttt{expand.grid()} can be
vectors, factors or a list containing these. Thus,

\begin{Shaded}
\begin{Highlighting}[]
\NormalTok{list_args <-}\StringTok{ }\KeywordTok{rep}\NormalTok{(}\KeywordTok{list}\NormalTok{(all_students), }\DecValTok{5}\NormalTok{)}
\end{Highlighting}
\end{Shaded}

could be convenient to use in your answer.

\hypertarget{all-pincodes}{%
\subsection{1.2 All Pincodes}\label{all-pincodes}}

Test whether you are fluent with \texttt{R} already, this exercise
should not take you more than 10 minutes at an exam.

Create a vector that contains all possible unique PIN codes of bank
cards that consist of 4 numbers. The first element of the vector is
\texttt{"0000"} and the last element of the vector is \texttt{"9999"}.

Hint: You may want to use the \texttt{paste0()} function with the
\texttt{collapse\ =\ ""} argument in combination with an implicit loop
(\texttt{{[}*{]}pply}), or an explicit loop
(\texttt{for\ (elem\ in\ vector)\{\}}.

\hypertarget{lady-tasting-tea-to-the-max.}{%
\subsection{1.3 Lady tasting tea to the
max.}\label{lady-tasting-tea-to-the-max.}}

Perform a permutation test for Dr.~Muriel Bristol's exhaustive tea
tasting experiment where she had 800 cups of the ``Milk first'' cups
correct.

We generated the data with the following \emph{R} code:

\begin{Shaded}
\begin{Highlighting}[]
\KeywordTok{set.seed}\NormalTok{(}\DecValTok{171123}\NormalTok{)}
\NormalTok{Truth <-}\StringTok{ }\KeywordTok{rep}\NormalTok{(}\KeywordTok{c}\NormalTok{(}\StringTok{"Milk"}\NormalTok{, }\StringTok{"Tea"}\NormalTok{), }\DataTypeTok{each =} \DecValTok{1000}\NormalTok{)}
\NormalTok{choice_Lady <-}\StringTok{ }\NormalTok{Truth}
\NormalTok{MilkyChange_idx <-}\StringTok{ }\KeywordTok{sample}\NormalTok{(}\DecValTok{1}\OperatorTok{:}\DecValTok{1000}\NormalTok{, }\DecValTok{200}\NormalTok{)}
\NormalTok{TeaChange_idx <-}\StringTok{ }\KeywordTok{sample}\NormalTok{(}\DecValTok{1001}\OperatorTok{:}\DecValTok{2000}\NormalTok{, }\DecValTok{200}\NormalTok{)}
\NormalTok{choice_Lady[MilkyChange_idx] <-}\StringTok{ "Tea"}
\NormalTok{choice_Lady[TeaChange_idx] <-}\StringTok{ "Milk"}
\end{Highlighting}
\end{Shaded}

and we use the function:

\begin{Shaded}
\begin{Highlighting}[]
\NormalTok{GetNrSuccesses <-}\StringTok{ }\ControlFlowTok{function}\NormalTok{(choice, truth) \{}
  \KeywordTok{sum}\NormalTok{(}\StringTok{"Milk"} \OperatorTok{==}\StringTok{ }\NormalTok{choice }\OperatorTok{&}\StringTok{ "Milk"} \OperatorTok{==}\StringTok{ }\NormalTok{truth) }\CommentTok{# R recycling behavior trick}
\NormalTok{\}}
\NormalTok{t_stat <-}\StringTok{ }\KeywordTok{GetNrSuccesses}\NormalTok{(choice_Lady, Truth)}
\NormalTok{t_stat}
\end{Highlighting}
\end{Shaded}

\begin{verbatim}
## [1] 800
\end{verbatim}

Code your own permutation test with the above ingredients. What do you
think Sir Ronald Fisher would have concluded from this experiment?

\hypertarget{exercises-part-2}{%
\section{Exercises part 2}\label{exercises-part-2}}

\hypertarget{a-permutation-test-on-students-sleep-data}{%
\subsection{2.1 A Permutation test on Student's sleep
data}\label{a-permutation-test-on-students-sleep-data}}

Create a function with which you can perform a permutation test instead
of the following two-samples t.test:

\begin{Shaded}
\begin{Highlighting}[]
\NormalTok{t_stat <-}\StringTok{ }\KeywordTok{t.test}\NormalTok{(extra }\OperatorTok{~}\StringTok{ }\NormalTok{group, }\DataTypeTok{data =}\NormalTok{ sleep)}
\end{Highlighting}
\end{Shaded}

\emph{Note:} stick to the example of the helpfile of the
\texttt{t.test()} function in \texttt{R} (\texttt{example(t.test})).
Thus, assume we have two independent groups being measured on their
\texttt{extra} hours of sleep. We supposedly have 2 groups of each 10
persons in the data.

\hypertarget{two-samples-permutation-tests-of-the-trimmed-mean}{%
\subsection{2.2 Two samples permutation tests of the trimmed
mean}\label{two-samples-permutation-tests-of-the-trimmed-mean}}

In this exercise we create a test statistic \(T\) to test whether the
``20\% trimmed mean" of the samples \(X_1,...,X_m\) and those of
\(Y_1,...,Y_n\) come from populations where the mean is the same.

The data example:

\begin{Shaded}
\begin{Highlighting}[]
\KeywordTok{set.seed}\NormalTok{(}\DecValTok{160929}\NormalTok{)}
\NormalTok{mu <-}\StringTok{ }\FloatTok{0.5}
\NormalTok{x <-}\StringTok{ }\KeywordTok{rnorm}\NormalTok{(}\DecValTok{8}\NormalTok{) }\CommentTok{# sampling distr F}
\NormalTok{y <-}\StringTok{ }\KeywordTok{rnorm}\NormalTok{(}\DecValTok{12}\NormalTok{, mu, }\DecValTok{2}\NormalTok{) }\CommentTok{# sampling distr G}
\end{Highlighting}
\end{Shaded}

The test statistic:

\begin{Shaded}
\begin{Highlighting}[]
\NormalTok{Get20TrimmedMean <-}\StringTok{ }\ControlFlowTok{function}\NormalTok{(x,y) \{}
\NormalTok{  out <-}\StringTok{ }\KeywordTok{mean}\NormalTok{(x, }\DataTypeTok{trim =} \FloatTok{0.2}\NormalTok{) }\OperatorTok{-}\StringTok{ }\KeywordTok{mean}\NormalTok{(y, }\DataTypeTok{trim =} \FloatTok{0.2}\NormalTok{) }
  \KeywordTok{return}\NormalTok{(out)}
\NormalTok{\}}
\NormalTok{t_obs <-}\StringTok{ }\KeywordTok{Get20TrimmedMean}\NormalTok{(x,y)}
\NormalTok{t_obs}
\end{Highlighting}
\end{Shaded}

\begin{verbatim}
## [1] -1.089706
\end{verbatim}

\hypertarget{a}{%
\subsubsection{a}\label{a}}

How many times does each \textbf{combination} of the \(m = 8\) samples
out \(N = n + m = 20\) occur in the \texttt{factorial(N)} possible
\textbf{permutations}?

\hypertarget{b}{%
\subsubsection{b}\label{b}}

Can you create the whole ``sample space'' for all combinations of
\(m= 8\) combinations out of \(N= 20\)? Store it into an object. Use
\texttt{utils::combn()}.

\hypertarget{c}{%
\subsubsection{c}\label{c}}

Perform a two-sided permutation test for the difference between the 20\%
trimmed of \(X_1,...,X_m\) and \(Y_1,...,Y_n\).

Use all samples from the complete sample space (that does not contain
any doubles). You have computed this sample space in \textbf{b}. Thus,
\texttt{B\ \textless{}-\ choose(m\ +\ n,\ m)}. Using the same data
everytime, and without setting a seed, would your \(p\)-value be the
same every time you run this test?

\emph{Note: the test may run a while on your computer!}

\hypertarget{d}{%
\subsubsection{d}\label{d}}

Perform the two-sided permutation test again, but this time you do not
need to use the whole sample space variable \texttt{all\_combs} from
subtask \textbf{b}. Instead, use \(B = 1000\) random permutations, e.g.
\texttt{sample(1:N,\ m)}. Why is your p-value different from \textbf{c}?

\hypertarget{e}{%
\subsubsection{e}\label{e}}

Would you know of a study / experiment to check whether on average the
previous two test are the same? You don't need to programme this in
\texttt{R}, though for the assignment this could perhaps be a
task\ldots{}\ldots{}

\hypertarget{paired-two-sample-permutation-test}{%
\subsection{2.3 Paired two-sample permutation
test}\label{paired-two-sample-permutation-test}}

The data:

\begin{Shaded}
\begin{Highlighting}[]
\KeywordTok{set.seed}\NormalTok{(}\DecValTok{190933}\NormalTok{)}
\NormalTok{mu <-}\StringTok{ }\FloatTok{1.75}
\NormalTok{x <-}\StringTok{ }\KeywordTok{rnorm}\NormalTok{(}\DecValTok{30}\NormalTok{)}
\NormalTok{y <-}\StringTok{ }\KeywordTok{rnorm}\NormalTok{(}\DecValTok{30}\NormalTok{, mu, }\DecValTok{2}\NormalTok{)}
\end{Highlighting}
\end{Shaded}

The test statistic (but now for paired testing\ldots{}):

\begin{Shaded}
\begin{Highlighting}[]
\NormalTok{Get20TrimmedMean <-}\StringTok{ }\ControlFlowTok{function}\NormalTok{(x,y) }\KeywordTok{mean}\NormalTok{(x, }\DataTypeTok{trim =} \FloatTok{0.2}\NormalTok{) }\OperatorTok{-}\StringTok{ }\KeywordTok{mean}\NormalTok{(y, }\DataTypeTok{trim =} \FloatTok{0.2}\NormalTok{) }
\NormalTok{t_obs <-}\StringTok{ }\KeywordTok{Get20TrimmedMean}\NormalTok{(x,y)}
\NormalTok{t_obs}
\end{Highlighting}
\end{Shaded}

\begin{verbatim}
## [1] -2.040649
\end{verbatim}

To perform a permutation test we need to swap the elements in \(X\) and
\(Y\). We have \texttt{2\^{}length(x)} possibilities for swapping the
elements in X and Y.

\hypertarget{a-1}{%
\subsubsection{a}\label{a-1}}

Create a function that can perform a permutation test (set \(B\) = 1000
as a default). Your function shoul output the \(p\)-value, the estimated
mean and the estimated standard error of your test statistic.

Start with the default permutation test where we just use

\texttt{swaps\ \textless{}-\ sample(c(0,\ 1),\ N/2,\ replace\ =\ TRUE)}

directly for each permutation. When \texttt{swaps{[}1{]}\ ==\ 1}, it
means that the first value for \(X\) and \(Y\) should be swapped, when
it \texttt{swaps{[}1{]}\ ==\ 0}, then we do not swap these values.

\hypertarget{b-1}{%
\subsubsection{b}\label{b-1}}

Note that the p-values in \textbf{a} can vary a lot (if you don't use
\texttt{set.seed}). Since there are ``only'' \texttt{2\^{}length(x)}
possibilties, it is highly likely that with \texttt{B\ =\ 1000} draws,
and with \texttt{sample(c(0,\ 1),\ length(x),\ replace\ =\ TRUE)} we
will get multiple swap vectors that may be the same.

The probability of one double swap vector is related
\href{https://en.wikipedia.org/wiki/Birthday_problem}{birthday problem}
formula:

\begin{Shaded}
\begin{Highlighting}[]
\NormalTok{B <-}\StringTok{ }\DecValTok{1000}
\DecValTok{1} \OperatorTok{-}\StringTok{ }\KeywordTok{prod}\NormalTok{(}\DecValTok{1} \OperatorTok{-}\StringTok{ }\DecValTok{1}\OperatorTok{:}\NormalTok{(B }\OperatorTok{-}\StringTok{ }\DecValTok{1}\NormalTok{)}\OperatorTok{/}\NormalTok{(}\DecValTok{2}\OperatorTok{^}\KeywordTok{length}\NormalTok{(x)))}
\end{Highlighting}
\end{Shaded}

\begin{verbatim}
## [1] 0.0004650876
\end{verbatim}

Hence, the probability is practically 1!

\emph{Perhaps} a practical solution to this variability problem of the
p-values in \textbf{a} would be to create a random swap matrix of \(B\)
rows (and N/2 columns) beforehand. Each row \(b\) (out of the \(B\)
rows) is a realization of
\texttt{sample(c(0,\ 1),\ length(x),\ replace\ =\ TRUE)}, but all rows
are unique.

To do so, first sample e.g. \(3B\) rows, and then check whether you can
reduce this ``swap'' matrix into \(B\) unique rows. If not, sample
\(4B\) rows, etc. The code we used:

\begin{Shaded}
\begin{Highlighting}[]
\CommentTok{# The swap matrix:}
\NormalTok{swaps <-}\StringTok{ }\KeywordTok{sample}\NormalTok{(}\KeywordTok{c}\NormalTok{(}\DecValTok{0}\NormalTok{, }\DecValTok{1}\NormalTok{), }\KeywordTok{length}\NormalTok{(x)}\OperatorTok{*}\NormalTok{(B}\OperatorTok{*}\DecValTok{3}\NormalTok{), }\DataTypeTok{replace =} \OtherTok{TRUE}\NormalTok{)}
\KeywordTok{dim}\NormalTok{(swaps) <-}\StringTok{ }\KeywordTok{c}\NormalTok{(B}\OperatorTok{*}\DecValTok{3}\NormalTok{, }\KeywordTok{length}\NormalTok{(x))}
\NormalTok{swaps <-}\StringTok{ }\KeywordTok{unique}\NormalTok{(swaps)[}\DecValTok{1}\OperatorTok{:}\NormalTok{B, ] }\CommentTok{# select only the unique rows}
\end{Highlighting}
\end{Shaded}

Code again the permutation test, but this time use a beforehand
specified swap matrix. Did it help to get a more robust (= stable)
\(p\)-value?

\hypertarget{c-1}{%
\subsubsection{c}\label{c-1}}

The p-value remains quite unstable in \textbf{b}. Code a permutation
test where we use all unique swap possibilities (the whole sample space,
without duplications). Thus, \(B = 4096\). What is the permutation test
\(p\)-value that we were trying to estimate in \textbf{a} and
\textbf{b}?

\emph{Hint: You may want to use the function \texttt{expand.grid()}}

\hypertarget{d-1}{%
\subsubsection{d}\label{d-1}}

What could be a reason that the variability in results of the
permutation test in \textbf{a} and \textbf{b} are impossible to test for
this data set?

\newpage

\hypertarget{self-study-permutations-in-genetics}{%
\section{self-study: Permutations in
Genetics}\label{self-study-permutations-in-genetics}}

\emph{by courtesy of Jelle Goeman}

Install the package \texttt{penalized}, and have a look at the nki70
data contained in that package. You should be able to run the following
code:

\begin{Shaded}
\begin{Highlighting}[]
\KeywordTok{library}\NormalTok{(penalized)}
\end{Highlighting}
\end{Shaded}

\begin{verbatim}
## Loading required package: survival
\end{verbatim}

\begin{verbatim}
## Welcome to penalized. For extended examples, see vignette("penalized").
\end{verbatim}

\begin{Shaded}
\begin{Highlighting}[]
\KeywordTok{data}\NormalTok{(nki70)}
\end{Highlighting}
\end{Shaded}

The nki70 data consists of gene expression data of 70 genes that are
prognostic for disease-free survival in breast cancer patients, the
so-called ``mammaprint'' signature, together with survival status and
some clinical covariates. These genes and covariates are measured for a
sample of 144 young breast cancer patients.

We are interested in association between gene expression and estrogen
receptor status (ER), which is itself an important predictor of breast
cancer prognosis.

Install the package \texttt{penalized}, and have a look at the nki70
data contained in that package. You should be able to run the following
code:

\begin{Shaded}
\begin{Highlighting}[]
\KeywordTok{library}\NormalTok{(penalized)}
\KeywordTok{data}\NormalTok{(nki70)}
\end{Highlighting}
\end{Shaded}

\hypertarget{permutations-in-genetics}{%
\subsection{2.4 Permutations in
Genetics}\label{permutations-in-genetics}}

\hypertarget{a-2}{%
\subsubsection{a}\label{a-2}}

Test whether the expression of the WISP1 gene is associated with ER
status. You can choose any test of your preference. In the model answers
we used a t-test.

\hypertarget{b-2}{%
\subsubsection{b}\label{b-2}}

Make a randomly permuted version of the ER status variable, and test for
association of the WISP1 gene with the permuted ER.

\hypertarget{c-2}{%
\subsubsection{c}\label{c-2}}

Repeat exercise (b) 100 times and make a histogram of the 100 resulting
test statistics.

\hypertarget{d-2}{%
\subsubsection{d}\label{d-2}}

Calculate a permutation p-value. Hint: two-sided test: remember to take
absolute values

\hypertarget{e-1}{%
\subsubsection{e}\label{e-1}}

Make a histogram of the p-values for the 100 permutations.

\hypertarget{f}{%
\subsubsection{f}\label{f}}

Test association with ER status for all 70 genes (columns 8 to 77).

\hypertarget{g}{%
\subsubsection{g}\label{g}}

Use Bonferroni to correct the results for multiple testing. How many
null hypotheses can be rejected at a type-I error rate of 0.05?

\hypertarget{multipe-testing-by-permutation}{%
\subsection{2.5 Multipe testing by
permutation}\label{multipe-testing-by-permutation}}

\hypertarget{a-3}{%
\subsubsection{a}\label{a-3}}

Calculate the minimum p-value over all 70 genes.

\hypertarget{b-3}{%
\subsubsection{b}\label{b-3}}

Permute ER status, test association of all genes with permuted ER status
and calculate the minimum p-value.

\hypertarget{c-3}{%
\subsubsection{c}\label{c-3}}

Repeat the permutation 100 times and make a histogram of the minimum
p-values.

\hypertarget{d-3}{%
\subsubsection{d}\label{d-3}}

Find the 0.05 quantile of the permutation distribution of minimum
p-values, and compare with the Bonferroni threshold.

\hypertarget{e-2}{%
\subsubsection{e}\label{e-2}}

How many hypotheses can be rejected based on the critical value found in
exercise (d)?

\hypertarget{f-1}{%
\subsubsection{f}\label{f-1}}

Repeat sub exercises 2.5c - 2.5e using only the hypotheses that could
not (yet) be rejected in exercise (e). How many new rejections are made?
And how many hypotheses in total?

\emph{Hint: The answers you should find with \texttt{R} are 3 and 34}

\hypertarget{g-1}{%
\subsubsection{g}\label{g-1}}

Optional: repeat exercise (f) until no further rejections occur. In this
way you've created a permuted version of the {[}holm's p-value
method{]}. See the \texttt{stats::p.adjust} function or check:

\emph{Holm, S. (1979). A simple sequentially rejective multiple test
procedure. Scandinavian Journal of Statistics 6, 65-70.}


\end{document}
